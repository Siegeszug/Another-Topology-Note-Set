\documentclass[12pt, twosided]{article}

\usepackage[letterpaper,bindingoffset=0in,%
            left=1in,right=1in,top=1in,bottom=1in,%
            footskip=.25in]{geometry}

\usepackage{mathtools}
\usepackage{graphicx}

\usepackage{setspace}
\setstretch{1.1}

\usepackage{amsmath}
\usepackage{amsfonts}
\usepackage{amsthm}
\usepackage{amssymb}
\usepackage{csquotes}
\usepackage{relsize}

\usepackage{tikz}
\usetikzlibrary{cd}
\usetikzlibrary{fit,shapes.geometric}
\tikzset{%  
    mdot/.style={draw, circle, fill=black},
    mset/.style={draw, ellipse, very thick},
}

\usepackage{hhline}
\usepackage{systeme}
\usepackage{mathrsfs}
\usepackage{hyperref}
\usepackage{mathtools}  
\usepackage{silence}
\usepackage{blkarray}
\usepackage{float}
\usepackage{framed}
\usepackage{array}
\usepackage{stmaryrd}
\usepackage{extarrows}
\usepackage{caption}
\captionsetup[figure]{labelfont={bf},name={Fig.},labelsep=period}

\theoremstyle{definition}
\newtheorem{df}{Definition}
\newtheorem{exa}{Example}
\newtheorem{ques}{Question}
\newtheorem{exr}{Exercise}
\newtheorem*{note}{Note}
\theoremstyle{plain}
\newtheorem{thm}{Theorem}
\newtheorem{prop}{Proposition}
\newtheorem{conj}{Conjecture}
\newtheorem{cor}{Corollary}
\newtheorem{lm}{Lemma}
\newtheorem*{fact}{Fact}
\newtheorem*{idea}{Idea}
\newtheorem*{clm}{Claim}
\newtheorem*{rmk}{Remark}
\usepackage[ruled]{algorithm2e}

\usepackage{ulem}
\makeatletter

\def\lf{\left\lfloor}   
\def\rf{\right\rfloor}
\def\lc{\left\lceil}   
\def\rc{\right\rceil}
\def\st{\text{ s.t. }}
\def\1{^{-1}}
\def\ind{\mathbf{1}}
\def\R{\mathbb{R}}
\def\Q{\mathbb{Q}}
\def\Z{\mathbb{Z}}
\def\C{\mathbb{C}}
\def\I{\mathbb{I}}
\def\N{\mathbb{N}}
\def\F{\mathbb{F}}
\def\A{\mathbb{A}}
\def\Li{\text{Li}}
\def\th{^\text{th}}
\def\sp{\text{Sp}}
\def\opn{\left\{}
\def\cls{\right\}}
\def\Aut{\text{Aut}}
\def\PG{\text{PG}}
\def\GL{\text{GL}}
\def\PGL{\text{PGL}}
\def\Cov{\text{Cov}}
\def\Pack{\text{Pack}}
\def\PgamL{\text{P}\Gamma\text{L}}
\def\gamL{\Gamma\text{L}}
\def\cl{\text{cl}}
\def\stbar{\ \middle\vert\ }
\def\partdone{\hphantom{1} \hfill \(\triangle\)}
\def\s0{_0}
\def\s1{_1}
\def\s2{_2}
\def\id{\mathrm{id}}
\def\topn{\text{ open}}
\def\Bd{\text{Bd }}
\renewcommand{\P}{\mathbb{P}}
\newcommand{\leg}[2]{\left( \frac{#1}{#2} \right)}
\renewcommand*\env@matrix[1][*\c@MaxMatrixCols c]{%
   \hskip -\arraycolsep
   \let\@ifnextchar\new@ifnextchar
   \array{#1}}
\makeatother

% These two lines suppress the warning generated 
% by amsmath for overwriting the choose command  
% because it's annoying. This probably has unint-
% ended ramifications somewhere else, but I'm too
% lazy to actually figure that out, so we'll cro-
% ss that bridge when we come to it lol.
\renewcommand{\choose}[2]{\left( {#1 \atop #2} \right)}
\WarningFilter{amsmath}{Foreign command} 

\renewcommand{\mod}[1]{\ (\mathrm{mod}\ #1)}
\renewcommand{\vec}[1]{\mathbf{#1}}

\let\oldprime\prime
\def\prime{^\oldprime}

\usepackage{float}
\restylefloat{figure}

\usepackage{cleveref}
\Crefname{thm}{Theorem}{Theorems}

% Comment commands for co-authors
\newcommand{\kmd}[1]{{\color{purple} #1}}

\newcolumntype{L}{>{$}l<{$}}
% Bib matter
\let\oldepsilon\epsilon
\def\epsilon{\varepsilon}

\let\oldphi\phi
\def\phi{\varphi}

%%% Local Variables:
%%% mode: plain-tex
%%% TeX-master: t
%%% End:

\graphicspath{{./img/}}

\begin{document}
\noindent \textbf{Math 171} \hfill \textbf{Professor Sebastian Bozlee} \\
\textbf{Scribed by: Kyle Dituro} \hfill \textbf{February 14, 2023}\hrule
\vspace{.2in}

During the example last week, we found that it can be useful to consider the preimage of a function when looking at continuous functions. This leads us into the following major theorem.

\begin{thm}
  Let \(f: \R^n \to \R^m\). Then \(f\) is continuous iff for each open subset \(U \subseteq \R^m\) the subset \(f\1(U)\subseteq \R^n\) is open.
\end{thm}
\begin{proof}
  \begin{enumerate}
  \item [(\(\Rightarrow\))] Suppose \(f: \R^n \to \R^m\) is continuous. Let \(U \subseteq \R^m\) be an open set. Let \(\vec{x} \in f\1(U)\). Since \(U\) is open, \(f(\vec{x}) \in U\), there \(\exists \epsilon > 0\) such that \(B(f(\vec{x}), \epsilon) \subseteq U\).

    Since \(f\) is continuous, \(\exists \delta > 0\) such that \(\forall \vec{y} \in \R^n\), with \(|\vec{x} - \vec{y}| < \delta\), \(|f(\vec{x}) - f(\vec{y})| < \epsilon\). In other words, for all \(\vec{y} \in B(\vec{x}, \delta) \subseteq f\1(U)\) therefore \(f\1(U)\) is open.
  \item [(\(\Leftarrow\))] Assume that \(f: \R^n \to \R^m\) has the property that \(U \subseteq \R^m\) is open \(\Rightarrow f\1(U) \subseteq \R^n\). Now let \(\vec{x} \in \R^n\), \(\epsilon > 0\).

    Now \(U = B(f(\vec{x}), \epsilon) \subseteq \R^m\) is an open subset, so \(f\1(U) \subseteq \R^n)\) is open. So since \(f\1(U)\) is open, and since \(\vec{x} \in f\1(U)\), so by the definition of openness, there exists a real number \(\delta > 0\) such that \(B(\vec{x}, \delta) \subseteq f\1(U)\). In other words, for all \(\vec{y} \in \R^n\) with \(|\vec{x} - \vec{y}| < \delta\), \(f(\vec{y}) \in U\).

    Therefore \(\forall \vec{y} \in \R^n\) with \(|\vec{x} - \vec{y}| < \delta\), \(|f(\vec{x}) - f(\vec{y})| < \epsilon\). Therefore \(f\) is continuous.
  \end{enumerate}
\end{proof}

We of course also know that \(f\1\) preserves all the standard set-theoretic operations.

\begin{enumerate}
\item \(f\1(\bigcup U_i) = \bigcup f\1(U_i)\)
\item \(f\1(\bigcap U_i) = \bigcap f\1(U_1)\)
\item \(f\1(U^c) = (f\1(U))^c\)
\end{enumerate}

\begin{prop}
  \begin{enumerate}
  \item \(\emptyset, \R^n \subseteq \R^n\) are open in \(\R^n\)
  \item If \({U}_{i\in I}\) is a collection of open subsets of \(\R^n\), then their union is also open.
  \item If \(U_1,\ldots, U_n\) are open subsets of \(\R^n\) then \(\cap_{i=1}^nU_i\)is also an open subset of \(\R^n\).
  \end{enumerate}
\end{prop}
\begin{proof}
  \begin{enumerate}
  \item The empty set is open vacuously.
    \(\forall \vec{x} \in \R^n\), \(B(\vec{x}, 1) \subseteq \R^n\). 
  \item Unions come somewhat naturally. Namely let \(\vec{x} \in \cup_{i \in I}U_i\), then there is an index \(j \in I\) such that \(\vec{x} \in U_j\). Then since \(U_j\) is open, there is some \(\epsilon > 0\) such that \(B(\vec{x}, \epsilon) \subseteq U_j \subseteq \cup_{i \in I} U_i\). Therefore \(\cup_{i \in I} U_i\) is open.
  \item Intersections are a little bit more of a trick. Namely suppose that \(x \in \cap_{i = 1}^n U_i\). Then \(x \in U_i\)for each \(i\), so there exists a positive real number \(\epsilon_i > 0\)  for each \(i = 1, \ldots, n\) such that \(B(\vec{x}, \epsilon_i) \subseteq U_i\) for each \(i = 1, \ldots, n\). Let \(\epsilon = \min\opn \epsilon_1, \ldots, \epsilon_n \cls\). Then \(B(\vec{x}, \epsilon) \subseteq B(\vec{x}, \epsilon_i) \subseteq U_i\) for all \(i\), so \(B(\vec{x}, \epsilon) \subseteq \cap_{i = 1}^n U_i\).
  \end{enumerate}
\end{proof}
It aught to be suspicious that we can only do finite intersections... why is that?
\begin{exa}
  Set \(U_n = B(\vec{x}, \frac{1}{n})\) for each \(n \in \Z_{>0}\).
  Then \(\cap_{n = 1}^{\infty} U_n = \opn \vec{0} \cls\), which is not open.
  \begin{figure}[h]
    \centering
    
    \caption[Concentric Circles]{We really can only do finite intersections.}
    \label{fig:FinInt}
  \end{figure}
\end{exa}

\begin{prop}
  Let \(Z \subseteq \R^n\) then \(Z\) is a closed subset of \(\R^n\) iff \(Z^c\) is an open subset of \(\R^n\).
\end{prop}
\begin{proof}
  \begin{enumerate}
  \item [(\(\Rightarrow\))]. Suppose that \(Z\) is closed. Then let \(\vec{x} \in Z^c\). Then \(\vec{x}\) is not a limit point of \(Z\) so \(\exists \epsilon > 0\) such that \(B(\vec{x}, \epsilon) \cap Z) = \emptyset.\) Then \(B(\vec{x}, \epsilon) \subseteq Z^c\) so \(Z^c\) is open.
  \item [(\(\Leftarrow\))] Suppose that \(Z^c\)  is open. Then let \(\vec{x} \in \R^n\) be a limit point of \(Z\). Then for all \(\epsilon > 0\), \(B(\vec{x}, \epsilon) \cap Z \neq \emptyset\). So \(\forall \epsilon > 0\), \(B(\vec{x}, \epsilon) \not\subseteq Z^c\). This implies that \(\vec{x} \not\in Z^c\) since \(Z^c\) is open. Thus \(\vec{x} \in Z\), so \(Z\) is closed.
  \end{enumerate}
\end{proof}
% ================================
\begin{df}
  Let \(X\) be a set. A \textbf{topology on} \(X\) is a set \(\tau\) of subsets 
of \(X\) such that:
\begin{enumerate}
  \item \(\emptyset, X \in \tau\)
  \item If \(U_i \in \tau\) of each \(i \in I\) then \(\cup_{i \in I} U_i \in \tau\)
  \item If \(U_1, \ldots, U_n \in \tau\), then \(\cap_{i = 1}^n U_i \in \tau\).
  \end{enumerate}
    A \textbf{topological space} is a pair \((X, \tau)\) where \(X\) is a set and \(\tau\) is a topology on \(X\)., Then the elements of \(\tau\) are called the \textbf{open subsets} of \(X\).

\end{df}
%=========================
\begin{exa}
  \(\R^n\) with the usual definition of open sets is a topological space.
\end{exa}

\begin{exa}
  We can define a topological space with only finitely many points. For example
  \begin{align*}
    X &= \opn 1, 2 \cls \\
    \tau &= \opn \emptyset, \opn 1, 2 \cls \cls \quad &\text{(indiscreet topology)}\\
    \tau_1 &= \opn \emptyset, \opn 1 \cls, \opn 1, 2 \cls \cls \quad &\text{(Sierpinski space)} \\
    \tau_2 &= \mathcal{P}(X) \quad &\text{(The discrete topology)}
  \end{align*}
\end{exa}

\begin{df}
  If \((X, \tau_X), (Y, \tau_Y)\) are topological spaces, then a function \(f: X \to Y\) is said to be \textbf{continuous} if \(f\1(U)\) is open in \(Y\) whenever \(U\) is open in \(X\).
\end{df}

\begin{exa}
  What are the continuous functions from  \(S\) (the Sierpinski space) to \(S\).

  There are four functions:
  \begin{center}
    \begin{tabular}[h]{c|cc|c}
      \(x\) & 1 & 2 & cont?\\
      \hline
      \(f_1(x)\) & 1 & 2 & Yes\\
      \(f_2(x)\) & 1 & 1 & Yes\\
      \(f_3(x)\) & 2 & 1 & \(f_3\1({1}) = {2}\)\\
      \(f_4(x)\) & 2 & 2 & Yes
    \end{tabular}
  \end{center}
\end{exa}
\end{document}
%%% Local Variables:
%%% mode: latex
%%% TeX-master: t
%%% End:

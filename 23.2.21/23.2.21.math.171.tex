\documentclass[12pt, twosided]{article}

\usepackage[letterpaper,bindingoffset=0in,%
            left=1in,right=1in,top=1in,bottom=1in,%
            footskip=.25in]{geometry}

\usepackage{mathtools}
\usepackage{graphicx}

\usepackage{setspace}
\setstretch{1.1}

\usepackage{amsmath}
\usepackage{amsfonts}
\usepackage{amsthm}
\usepackage{amssymb}
\usepackage{csquotes}
\usepackage{relsize}

\usepackage{tikz}
\usetikzlibrary{cd}
\usetikzlibrary{fit,shapes.geometric}
\tikzset{%  
    mdot/.style={draw, circle, fill=black},
    mset/.style={draw, ellipse, very thick},
}

\usepackage{hhline}
\usepackage{systeme}
\usepackage{mathrsfs}
\usepackage{hyperref}
\usepackage{mathtools}  
\usepackage{silence}
\usepackage{blkarray}
\usepackage{float}
\usepackage{framed}
\usepackage{array}
\usepackage{stmaryrd}
\usepackage{extarrows}
\usepackage{caption}
\captionsetup[figure]{labelfont={bf},name={Fig.},labelsep=period}

\theoremstyle{definition}
\newtheorem{df}{Definition}
\newtheorem{exa}{Example}
\newtheorem{ques}{Question}
\newtheorem{exr}{Exercise}
\newtheorem*{note}{Note}
\theoremstyle{plain}
\newtheorem{thm}{Theorem}
\newtheorem{prop}{Proposition}
\newtheorem{conj}{Conjecture}
\newtheorem{cor}{Corollary}
\newtheorem{lm}{Lemma}
\newtheorem*{fact}{Fact}
\newtheorem*{idea}{Idea}
\newtheorem*{clm}{Claim}
\newtheorem*{rmk}{Remark}
\usepackage[ruled]{algorithm2e}

\usepackage{ulem}
\makeatletter

\def\lf{\left\lfloor}   
\def\rf{\right\rfloor}
\def\lc{\left\lceil}   
\def\rc{\right\rceil}
\def\st{\text{ s.t. }}
\def\1{^{-1}}
\def\ind{\mathbf{1}}
\def\R{\mathbb{R}}
\def\Q{\mathbb{Q}}
\def\Z{\mathbb{Z}}
\def\C{\mathbb{C}}
\def\I{\mathbb{I}}
\def\N{\mathbb{N}}
\def\F{\mathbb{F}}
\def\A{\mathbb{A}}
\def\Li{\text{Li}}
\def\th{^\text{th}}
\def\sp{\text{Sp}}
\def\opn{\left\{}
\def\cls{\right\}}
\def\Aut{\text{Aut}}
\def\PG{\text{PG}}
\def\GL{\text{GL}}
\def\PGL{\text{PGL}}
\def\Cov{\text{Cov}}
\def\Pack{\text{Pack}}
\def\PgamL{\text{P}\Gamma\text{L}}
\def\gamL{\Gamma\text{L}}
\def\cl{\text{cl}}
\def\stbar{\ \middle\vert\ }
\def\partdone{\hphantom{1} \hfill \(\triangle\)}
\def\s0{_0}
\def\s1{_1}
\def\s2{_2}
\def\id{\mathrm{id}}
\def\topn{\text{ open}}
\def\Bd{\text{Bd }}
\renewcommand{\P}{\mathbb{P}}
\newcommand{\leg}[2]{\left( \frac{#1}{#2} \right)}
\renewcommand*\env@matrix[1][*\c@MaxMatrixCols c]{%
   \hskip -\arraycolsep
   \let\@ifnextchar\new@ifnextchar
   \array{#1}}
\makeatother

% These two lines suppress the warning generated 
% by amsmath for overwriting the choose command  
% because it's annoying. This probably has unint-
% ended ramifications somewhere else, but I'm too
% lazy to actually figure that out, so we'll cro-
% ss that bridge when we come to it lol.
\renewcommand{\choose}[2]{\left( {#1 \atop #2} \right)}
\WarningFilter{amsmath}{Foreign command} 

\renewcommand{\mod}[1]{\ (\mathrm{mod}\ #1)}
\renewcommand{\vec}[1]{\mathbf{#1}}

\let\oldprime\prime
\def\prime{^\oldprime}

\usepackage{float}
\restylefloat{figure}

\usepackage{cleveref}
\Crefname{thm}{Theorem}{Theorems}

% Comment commands for co-authors
\newcommand{\kmd}[1]{{\color{purple} #1}}

\newcolumntype{L}{>{$}l<{$}}
% Bib matter
\let\oldepsilon\epsilon
\def\epsilon{\varepsilon}

\let\oldphi\phi
\def\phi{\varphi}

%%% Local Variables:
%%% mode: plain-tex
%%% TeX-master: t
%%% End:

\graphicspath{{./img/}}

\begin{document}
\noindent \textbf{Math 171} \hfill \textbf{Professor Sebastian Bozlee} \\
\textbf{Scribed by: Kyle Dituro} \hfill \textbf{February 21, 2023}\hrule
\vspace{.2in}

We begin with a definition:

\begin{df}
  Let \(X\) be a topological space. We say that \(Z \subseteq X\) is \textbf{closed} if \(Z^c \subseteq X\) is open in \(X\).
\end{df}

\begin{prop}
  Suppose that \(X\) is a topological space. Then:
  \begin{enumerate}
  \item \(X, \emptyset\) are closed subsets of \(X\).
  \item If \(Z_i \subset X\) for each \(i \in I\) is a closed subset of \(X\), then so is \(\bigcap_{i \in I}Z_i \subseteq X\) is closed is \(X\) as well.
  \item If \(Z_1, \ldots, Z_n\) are closed subsets of \(X\), then their union \(\bigcup_{i = 1}^nZ_i \subseteq X\) is as well.
  \end{enumerate}
\end{prop}
\begin{proof}
  We will show that this follows directly from the three tenants of what it means to be a topology.
  \begin{enumerate}
  \item Notice that \(\emptyset = X^c\) and likewise \(X = \emptyset^c\), so \(X\) and \(\emptyset\) are trivially clopen.
  \item For this and the following part, we will leverage De Morgan's laws. Let \(Z_i \subseteq X\) be a closed subset for each \(i \in I\) then
    \begin{align*}
      \bigcap_{i\in I}Z_i &= \left(\bigcap_{i \in I}Z_i^c\right)^c \\
                          &= \left(\bigcup_{i \in I} Z_{i}^c \right)^c
    \end{align*}
    But we know that \(Z_i^c\) is open, so this union is an open set, so its complement is closed.
  \item Let \(Z_i\), \(i = 1, \ldots, n\) be a closed collection of subsets of \(X\). Then again:

    \begin{align*}
      \bigcup_{i=1}^n Z_{i} &= \left(\bigcup_{i=1}^nZ_i^c\right)^c \\
                            &=\left(\bigcap_{i=1}^nZ_i^c \right)^c
    \end{align*}

    So we have the result again in the same way.
  \end{enumerate}
\end{proof}
\begin{idea}
  Any statement made in terms of open sets can be rephrased in terms of an equivalent statement about closed sets.
\end{idea}
\begin{fact}
  A topological space can be defined by specifying a collection of closed sets and showing that they satisfy the equivalent closed set definition of a topology.
\end{fact}
\begin{prop}
  If \(X,Y\) are topological spaces, then a function \(f: X \to Y\) s continuous iff whenever \(Z \subseteq Y\) is closed in \(Y\), \(f\1(Z)\) is a closed subset of \(X\).
\end{prop}
\begin{proof}
  \begin{enumerate}
  \item [(\(\Rightarrow\))] Assume \(f: X \to Y\) is continuous. Let \(Z \subseteq Y\)  be a closed subset. Then \(f\1(Z^c)^c\) is also closed. But \(Z^c\) is open, so \(f\1(Z^c)\) is open, and thus \(f\1(Z^c)^c\) is closed.
  \item [(\(\Leftarrow\))] Assume that pre-images of closed subsets are closed. So let \(U \subseteq Y\) be an open subset. Then \(f\1(U) = f\1(U^c)^c\), and the same thing happens again. \(f\1(U^c)\) is the pre-image of a closed set, which is closed, and so its complement is open. And the result falls out.
  \end{enumerate}
\end{proof}

\begin{df}
  A polynomial in \(n\) variable is a function \(f: \R^n \to \R\).
  \begin{align*}
    f(x_1 \ldots x_n) = \sum c_Ix^I \\
  \end{align*}

  The \textbf{Zero Locus} of a set of polynomials \(\{f_i\}_{i\in I}\)is the subset
  \begin{align*}
    Z(\{f_i\}) = \opn (x_1, \ldots, x_n) \in \R^n \middle\vert f_i(x_1 \ldots x_n) = 0 \cls.
  \end{align*}
\end{df}

For example: \(f(x, y) = y - x^2\):
\(Z(f)\) should be a parabola.

\begin{df}
  The zero locus of a set of polynomials is called an \textbf{algebraic variety}. The \textbf{Zariski} topology on \(\R^n\) is the topology on \(\R^n\) in which the algebraic varieties are the closed subsets.
\end{df}

\begin{df}
  Given a topological space \((X, \tau_X)\) and a subset \(Y \subseteq X\), the \textbf{subspace topology on} \(Y\) is the topology \[\tau_Y = \opn U \cap Y \middle\vert U \in \tau_X \cls\]
\end{df}
This is possibly the most important example that we will give.

\begin{exa}
  \(\{0, 1\} \subseteq \R\). Then \(\tau_Y = \{\emptyset, \{1\}, \{0\}, \{0, 1\}\}\) where \(\tau_Y\) is the subspace topology of the usual topology on \(\R\). This gives us a reason to call the discrete topology what it is...
\end{exa}

\begin{prop}
  Let \((X, \tau_X)\) a topological space, \((Y, \tau_Y)\) a subset of \(X\) with the subspace ``topology''. Then \(\tau_Y\) is a topology.
\end{prop}
\begin{proof}
  \begin{enumerate}
  \item \(\emptyset_{\tau_y} = \emptyset_{\tau_x} \ cap Y\), and furthermore \(Y = Y \cap X\).
  \item Suppose that \(U_i \in \tau_y\) for each \(i \in I\). By the definition of \(\tau_y\), for each \(U_i\), \(\exists V_i \in \tau_x\) such that \(V_i \cap Y = U_i\). Then
    \begin{align*}
      \bigcup_{i\in I} U_i = \bigcup_{i \in I} (V_i \cap Y) = \left( \bigcup_{i \in I} V_i \right) \cap Y \in \tau_Y.
    \end{align*}
    And the same idea for the intersections. 
  \item Let \(U_1 \ldots U_n \in \tau_Y\). Then we can find \(V_1, \ldots, V_n \in \tau_X\) such that \(U_i = V_i \cap Y\) for each \(i = 1 \ldots n\). Then
    \begin{align*}
      \bigcap_{i = 1}^n U_i &= \bigcap_{i =1}^n (V_i \cap Y) \\
                            &= \left(\bigcap_{i = 1}^n V_i\right) \cap Y \in \tau_Y
    \end{align*}
  \end{enumerate}
  So we really do have a topology.
\end{proof}

Note that it is actually important to do these intersections, since open subsets of a subset might \textbf{NOT} be open in the larger space.
\end{document}
%%% Local Variables:
%%% mode: latex
%%% TeX-master: t
%%% End:

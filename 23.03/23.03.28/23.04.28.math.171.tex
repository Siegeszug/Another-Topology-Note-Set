\documentclass[12pt, twosided]{article}
\usepackage[letterpaper,bindingoffset=0in,%
            left=1in,right=1in,top=1in,bottom=1in,%
            footskip=.25in]{geometry}

\usepackage{mathtools}
\usepackage{graphicx}

\usepackage{setspace}
\setstretch{1.1}

\usepackage{amsmath}
\usepackage{amsfonts}
\usepackage{amsthm}
\usepackage{amssymb}
\usepackage{csquotes}
\usepackage{relsize}

\usepackage{tikz}
\usetikzlibrary{cd}
\usetikzlibrary{fit,shapes.geometric}
\tikzset{%  
    mdot/.style={draw, circle, fill=black},
    mset/.style={draw, ellipse, very thick},
}

\usepackage{hhline}
\usepackage{systeme}
\usepackage{mathrsfs}
\usepackage{hyperref}
\usepackage{mathtools}  
\usepackage{silence}
\usepackage{blkarray}
\usepackage{float}
\usepackage{framed}
\usepackage{array}
\usepackage{stmaryrd}
\usepackage{extarrows}
\usepackage{caption}
\captionsetup[figure]{labelfont={bf},name={Fig.},labelsep=period}

\theoremstyle{definition}
\newtheorem{df}{Definition}
\newtheorem{exa}{Example}
\newtheorem{ques}{Question}
\newtheorem{exr}{Exercise}
\newtheorem{prb}{Problem}
\newtheorem*{note}{Note}
\theoremstyle{plain}
\newtheorem{thm}{Theorem}
\newtheorem{prop}{Proposition}
\newtheorem{conj}{Conjecture}
\newtheorem{cor}{Corollary}
\newtheorem{lm}{Lemma}
\newtheorem*{fact}{Fact}
\newtheorem*{idea}{Idea}
\newtheorem*{clm}{Claim}
\newtheorem*{rmk}{Remark}
\usepackage[ruled]{algorithm2e}

\usepackage{ulem}
\makeatletter

\def\lf{\left\lfloor}   
\def\rf{\right\rfloor}
\def\lc{\left\lceil}   
\def\rc{\right\rceil}
\def\st{\text{ s.t. }}
\def\1{^{-1}}
\def\ind{\mathbf{1}}
\def\R{\mathbb{R}}
\def\Q{\mathbb{Q}}
\def\Z{\mathbb{Z}}
\def\C{\mathbb{C}}
\def\I{\mathbb{I}}
\def\N{\mathbb{N}}
\def\F{\mathbb{F}}
\def\A{\mathbb{A}}
\def\Li{\text{Li}}
\def\th{^\text{th}}
\def\sp{\text{Sp}}
\def\opn{\left\{}
\def\cls{\right\}}
\def\Aut{\text{Aut}}
\def\PG{\text{PG}}
\def\GL{\text{GL}}
\def\PGL{\text{PGL}}
\def\Cov{\text{Cov}}
\def\Pack{\text{Pack}}
\def\PgamL{\text{P}\Gamma\text{L}}
\def\gamL{\Gamma\text{L}}
\def\cl{\text{cl}}
\def\stbar{\ \middle\vert\ }
\def\partdone{\hphantom{1} \hfill \(\triangle\)}
\def\s0{_0}
\def\s1{_1}
\def\s2{_2}
\def\id{\mathrm{id}}
\def\topn{\text{ open}}
\def\Bd{\text{Bd }}
\def\nope{\(\longrightarrow\!\!\longleftarrow\)}
\def\stt{\(^{\text{st}}\ \)}
\def\tht{\(^{\text{th}}\ \)}
\def\ndt{\(^{\text{nd}}\ \)}
\renewcommand{\P}{\mathbb{P}}
\newcommand{\leg}[2]{\left( \frac{#1}{#2} \right)}

\renewcommand*\env@matrix[1][*\c@MaxMatrixCols c]{%
   \hskip -\arraycolsep
   \let\@ifnextchar\new@ifnextchar
   \array{#1}}
\makeatother

% These two lines suppress the warning generated 
% by amsmath for overwriting the choose command  
% because it's annoying. This probably has unint-
% ended ramifications somewhere else, but I'm too
% lazy to actually figure that out, so we'll cro-
% ss that bridge when we come to it lol.
\renewcommand{\choose}[2]{\left( {#1 \atop #2} \right)}
\WarningFilter{amsmath}{Foreign command} 

\renewcommand{\mod}[1]{\ (\mathrm{mod}\ #1)}
\renewcommand{\vec}[1]{\mathbf{#1}}

\let\oldprime\prime
\def\prime{^\oldprime}

\usepackage{float}
\restylefloat{figure}

\usepackage{cleveref}
\Crefname{thm}{Theorem}{Theorems}

% Comment commands for co-authors
\newcommand{\kmd}[1]{{\color{purple} #1}}

\newcolumntype{L}{>{$}l<{$}}
% Bib matter
\let\oldepsilon\epsilon
\def\epsilon{\varepsilon}

\let\oldphi\phi
\def\phi{\varphi}

%%% Local Variables:
%%% mode: plain-tex
%%% TeX-master: t
%%% End:

\graphicspath{{./img/}}

\begin{document}
\noindent \textbf{Math 171} \hfill \textbf{Professor Sebastian Bozlee} \\
\textbf{Scribed by: Kyle Dituro} \hfill \textbf{March 28, 2023}\hrule
\vspace{.2in}

We will begin thinking about the idea of closure and interior:

\begin{df}
  Let \(X\) be a topological space, let \(A \subseteq X\). The \textbf{closure} \(\overline{A}\) of \(A\) is the intersection of all closed subsets of \(X\) containing \(A\).

  In other words, it is the smallest closed set containing \(A\)
\end{df}

\begin{df}
  The \textbf{interior} of \(A\), written \(\mathrm{Int}\ A\), (sometimes \(A^\circ\)) is the union of all open subsets of \(X\) contained in \(A\)
\end{df}

\begin{exa}
  \(A = (0, 1]\) in \(X = \R\) then
  \begin{align*}
    \overline{A} = [0, 1]\quad\quad A^\circ = (0, 1)
  \end{align*}

  Notice that the context of our closure is important. To see this, if \(A = (0, 1], X = (0, 2]\), then \(\overline{A} = (0, 1]\).
\end{exa}

\begin{lm}
  Let \(Y\) be a subspace of \(X\), (a subset of \(X\) given the subspace topology). Then a subset \(B \subseteq Y\) is closed iff there exists a closed set \(Z \subseteq X\) such that \(B = Z \cap Y\)
\end{lm}
\begin{proof}
  Observe that if \(U \subseteq X\) then

  \begin{align*}
    Y - (U \cap Y) = (X - U) \cap Y
  \end{align*}

  Then we have that \(B \subseteq Y\) is closed in \(Y\) iff \(Y - B\) is open in \(Y\) iff \(\exists U \subseteq X\) open such that \(Y -B = U \cap Y\) iff \(\exists U \subseteq X\) open such that \(B = Y - (U \cap Y)\) iff \(\exists U \subseteq X\) open such that \(B = (X -U) \cap Y\) iff \(\exists Z \subseteq X\) closed such that \(B = Z \cap Y\).
\end{proof}

\begin{thm}
  Let \(Y\) be a subspace of \(X\), \(A\) be a subset of \(Y\). Let \(\overline{A}\) be the closure of \(A\) in \(X\). Then the closure of \(A\) in \(Y\) is \(\overline{A} \cap Y\)
\end{thm}
\begin{proof}
  Let \(B\) be the closure of \(A\) in \(Y\). Then \(\overline{A}\) is closed in \(X\), so by the lemma, \(\overline{A} \cap Y\) is closed in \(Y\). On the other hand, \(\overline{A} \cap Y\) contains \(A\).

  Since \(B\) is in the intersection of the closed subsets of \(Y\) containing \(A\), \(B \subseteq \overline{A} \cap Y\).\partdone

  On the other hand, since \(B\) is closed in \(Y\), there exists a closed subset \(Z \subseteq X\) such that \(B = Z \cap Y\) by the lemma.
  This \(Z\) then is a closed subset of \(Z\) containing \(A\). Since \(\overline{A}\) is the intersection of all closed subsets of \(X\) containing \(A\), \(\overline{A} \subseteq Z\). Then

  \begin{align*}
    \overline{A} \cap Y \subseteq Z \cap Y = B 
  \end{align*}

  In such situations, \(\overline{A}\) will be the closure in \(X\). 
\end{proof}

We will now somehow demonstrate a relationship between the closure and the interior.

\begin{lm}
  Let \(A\) be a subset of a topological space \(X\). Then

  \begin{align*}
    \overline{A} = ((A^c)^\circ)^c \\
    A^\circ = (\overline{A^c})^c
  \end{align*}
\end{lm}

\begin{proof}
  \begin{align*}
    \overline{A} &=\left(\bigcap_{\substack{Z \subseteq X \mathrm{\ closed} \\ A \subseteq Z}} Z \right) \\
    A = \overline{A}^{cc} &=\left(\bigcap_{\substack{Z \subseteq X \mathrm{\ closed} \\ A \subseteq Z}} Z \right)^{cc} \\
                 &=\left(\bigcap_{\substack{Z \subseteq X \mathrm{\ closed} \\ A \subseteq Z}} Z^c \right)^c \\
                 &=\left(\bigcap_{\substack{Z^c \subseteq X \mathrm{\ open} \\ Z^c \subseteq A^c}} Z \right)^c \\
                 &=\left(\bigcap_{\substack{U \subseteq X \mathrm{\ open} \\ U \subseteq A^c}} U \right)^c \\
                 &= ((A^c)^\circ)^c
   \end{align*}
\end{proof}

  \begin{thm}
    Let \(A\) be a subset of the topological space \(X\). Then

    \begin{enumerate}
    \item \(x \in \overline{A}\) iff every open neighborhood \(U\) of \(x\) intersects \(A\). (where \(x\) is an adherent point).
    \item If the topology on \(X\) is generated by a basis \(\mathcal{B}\), then \(x \in \overline{A}\) iff every basic open neighborhood \(B\) of \(x\) intersects \(A\).
    \end{enumerate}
  \end{thm}

  \begin{proof}
    \begin{enumerate}
    \item Let's prove by contrapositive:

      \(x \not\in A\) iff There exists an open neighborhood of \(x\) such that \(U\) does not intersect \(A\).

      Now let's suppose \(x \not\in \overline{A}\) then \(U = X - \overline{A}\) is an open subset of \(X\) containing \(x\) and \(U \cap A = \emptyset\).

      Conversely, suppose that \(U\) is an open neighborhood of \(x\) not containing \(A\). then \(U^c\) is a closed subset of \(X\) containing \(A\). Then \(\overline{A} \subseteq U^c)\). Thus \(x \not\in \overline{A}\). \partdone
    \item Suppose now that all open neighborhoods \(U\) of \(x\) intersect \(A\). Then in particular this holds of the basic open neighborhoods, so each basic open neighborhood of \(x\) intersects \(A\).

      Conversely, suppose all basic open neighborhoods \(B\) of \(x\) intersect \(A\).

      If \(U\) is an open neighborhood of \(x\), then there exists a basis element \(B\) such that \(x\in B \subseteq U\). Then \(B \cap A \neq \emptyset\), so \(U \cap A \neq \emptyset\) as well.
    \end{enumerate}
  \end{proof} 

  \begin{df}
    A point \(x \in X\) is said to be a \textbf{limit point} of a subset \(A \subseteq X\) if each open neighborhood \(U\) of \(x\) intersects \(A\) in a point other than \(x\) (i.e. \((U - \{x\}) \cap A \neq \emptyset\)).
  \end{df}

  \begin{thm}
    Let \(A\) be a subset of a topological space \(X\). Write \(A\prime\) for the set of limit points of \(A\), then \(\overline{A} = A \cup A\prime\).
  \end{thm}

  \begin{proof}

    \((A \cup A\prime) \subseteq \overline{A}\) Let \(x \in A\prime\), then each open neighborhood of \(x\) intersects \(A\) in a point (other than \(x\)) so \(x \in \overline{A}\).

    If \(x \in A\) then \(x \in \overline{A}\) since \(A \subseteq \overline{A}\) by definition.
    
    So \(A \cup A\prime \subseteq \overline{A}\).

    For the reverse inclusion, let \(x \in A\). Then \(x \in A \cup A\prime\). If \(x \not\in A\) then any open neighborhood \(U\) of \(x\) intersects \(A\). Since \(x \not\in A\) this must be a point other than \(x\). So \(x\) is a limit point and \(x \in A\prime\). So \(x \in A \cup A\prime\). 
  \end{proof}

  \begin{cor}
    A subset \(A\) of a topological space \(X\) is closed iff it contains all of its limit points.
  \end{cor}
\end{document}
%%% Local Variables:
%%% mode: latex
%%% TeX-master: t
%%% End:

\documentclass[12pt, twosided]{article}

\usepackage[letterpaper,bindingoffset=0in,%
            left=1in,right=1in,top=1in,bottom=1in,%
            footskip=.25in]{geometry}

\usepackage{mathtools}
\usepackage{graphicx}

\usepackage{setspace}
\setstretch{1.1}

\usepackage{amsmath}
\usepackage{amsfonts}
\usepackage{amsthm}
\usepackage{amssymb}
\usepackage{csquotes}
\usepackage{relsize}

\usepackage{tikz}
\usetikzlibrary{cd}
\usetikzlibrary{fit,shapes.geometric}
\tikzset{%  
    mdot/.style={draw, circle, fill=black},
    mset/.style={draw, ellipse, very thick},
}

\usepackage{hhline}
\usepackage{systeme}
\usepackage{mathrsfs}
\usepackage{hyperref}
\usepackage{mathtools}  
\usepackage{silence}
\usepackage{blkarray}
\usepackage{float}
\usepackage{framed}
\usepackage{array}
\usepackage{stmaryrd}
\usepackage{extarrows}
\usepackage{caption}
\captionsetup[figure]{labelfont={bf},name={Fig.},labelsep=period}

\theoremstyle{definition}
\newtheorem{df}{Definition}
\newtheorem{exa}{Example}
\newtheorem{ques}{Question}
\newtheorem{exr}{Exercise}
\newtheorem*{note}{Note}
\theoremstyle{plain}
\newtheorem{thm}{Theorem}
\newtheorem{prop}{Proposition}
\newtheorem{conj}{Conjecture}
\newtheorem{cor}{Corollary}
\newtheorem{lm}{Lemma}
\newtheorem*{fact}{Fact}
\newtheorem*{idea}{Idea}
\newtheorem*{clm}{Claim}
\newtheorem*{rmk}{Remark}
\usepackage[ruled]{algorithm2e}

\usepackage{ulem}
\makeatletter

\def\lf{\left\lfloor}   
\def\rf{\right\rfloor}
\def\lc{\left\lceil}   
\def\rc{\right\rceil}
\def\st{\text{ s.t. }}
\def\1{^{-1}}
\def\ind{\mathbf{1}}
\def\R{\mathbb{R}}
\def\Q{\mathbb{Q}}
\def\Z{\mathbb{Z}}
\def\C{\mathbb{C}}
\def\I{\mathbb{I}}
\def\N{\mathbb{N}}
\def\F{\mathbb{F}}
\def\A{\mathbb{A}}
\def\Li{\text{Li}}
\def\th{^\text{th}}
\def\sp{\text{Sp}}
\def\opn{\left\{}
\def\cls{\right\}}
\def\Aut{\text{Aut}}
\def\PG{\text{PG}}
\def\GL{\text{GL}}
\def\PGL{\text{PGL}}
\def\Cov{\text{Cov}}
\def\Pack{\text{Pack}}
\def\PgamL{\text{P}\Gamma\text{L}}
\def\gamL{\Gamma\text{L}}
\def\cl{\text{cl}}
\def\stbar{\ \middle\vert\ }
\def\partdone{\hphantom{1} \hfill \(\triangle\)}
\def\s0{_0}
\def\s1{_1}
\def\s2{_2}
\def\id{\mathrm{id}}
\def\topn{\text{ open}}
\def\Bd{\text{Bd }}
\renewcommand{\P}{\mathbb{P}}
\newcommand{\leg}[2]{\left( \frac{#1}{#2} \right)}
\renewcommand*\env@matrix[1][*\c@MaxMatrixCols c]{%
   \hskip -\arraycolsep
   \let\@ifnextchar\new@ifnextchar
   \array{#1}}
\makeatother

% These two lines suppress the warning generated 
% by amsmath for overwriting the choose command  
% because it's annoying. This probably has unint-
% ended ramifications somewhere else, but I'm too
% lazy to actually figure that out, so we'll cro-
% ss that bridge when we come to it lol.
\renewcommand{\choose}[2]{\left( {#1 \atop #2} \right)}
\WarningFilter{amsmath}{Foreign command} 

\renewcommand{\mod}[1]{\ (\mathrm{mod}\ #1)}
\renewcommand{\vec}[1]{\mathbf{#1}}

\let\oldprime\prime
\def\prime{^\oldprime}

\usepackage{float}
\restylefloat{figure}

\usepackage{cleveref}
\Crefname{thm}{Theorem}{Theorems}

% Comment commands for co-authors
\newcommand{\kmd}[1]{{\color{purple} #1}}

\newcolumntype{L}{>{$}l<{$}}
% Bib matter
\let\oldepsilon\epsilon
\def\epsilon{\varepsilon}

\let\oldphi\phi
\def\phi{\varphi}

%%% Local Variables:
%%% mode: plain-tex
%%% TeX-master: t
%%% End:

\graphicspath{{./img/}}

\begin{document}
\noindent \textbf{Math 171} \hfill \textbf{Professor Sebastian Bozlee} \\
\textbf{Scribed by: Kyle Dituro} \hfill \textbf{March 1, 2023}\hrule
\vspace{.2in}

Herein we will outline some review content for the exam.

\begin{df}
  Let \(\opn X_i \cls_{i \in I}\) be a collection of sets indexed by \(I\). A set \(P\) together with functions \(\pi_i: P \to X_i\) for each \(i \in I\) is said to have the \textbf{universal property of the product} if, for any set \(Z\) and functions \(f_i: Z \to X_i\) for each \(i \in I\). Then there exists a unique function \(f: Z \to P\) such that
  \begin{center}
    \begin{tikzcd}
      Z \arrow[d, "f"] \arrow[dr, "f_i"] & \\
      P \arrow[r, "\pi_i"] & X_i
    \end{tikzcd}
  \end{center}
\end{df}

\begin{exa}
  A standard sort of product: Take \(X_1 = \R, X_2 = \R, P = \R^2\), and \(\pi_1: P \to X_1\), \((x, y) \mapsto x\), and \(\pi_2: P \to X_2\) taking \((x, y) \mapsto y\). then

  \begin{center}
    \begin{tikzcd}
      & Z \arrow[dotted, d, "\exists!f"] \arrow[dr, "f_2"] \arrow[dl, "f_1"] & \\
      X_1 & P \arrow[r,"\pi_2"] \arrow[l, "\pi_1"] & X_2 
    \end{tikzcd}
  \end{center}
  And in fact we can see that such an \(f\) does exist, as \(z \mapsto (f_1(z), f_2(z))\).
\end{exa}

\begin{exa}
  Here's a bit of a weirder one.
  \(X_1 = \R\), \(X_2 = \R\), \(P\prime = \R^2\), and the functions \(\pi_1\prime: P\prime \to X\) a projection on the first coordinate, and \(\pi_2\prime: P\prime \to X_2\) takes \((x, y) \mapsto x + y\).

  \begin{center}
    \begin{tikzcd}
      & Z \arrow[dotted, d, "\exists!f"] \arrow[dr, "f_2"] \arrow[dl, "f_1"] & \\
      X_1 & P\prime \arrow[l, "\pi_1\prime"] & X_2 \arrow[leftarrow, l,"\pi_2\prime"] 
    \end{tikzcd}
  \end{center}
  Notice that to get the \(f\) that we desire, we require that \(f(z) = (f_1(z), f_2(z) - f_1(z))\)
\end{exa}

\begin{thm}
  If \((P, \pi_i)\) has the universal property of the product and \((P\prime, \pi_i\prime)\) has universal property of the product, then there exists a unique bijection \(f: P \to P\prime\) such that \(\pi_i = \pi_i\prime\) for all \(i \in I\)

  \begin{center}
    \begin{tikzcd}
      P \arrow[d, "f"] \arrow[dr, "\pi_i"] & \\
      P\prime & X_i \arrow[leftarrow, l, "\pi_i\prime"]
    \end{tikzcd}
  \end{center}
\end{thm}

This theorem was proved in class, so we omit the proof here.

\begin{df}
  Let \(X\) be a set, \(\sim\) be an equivalence relation on \(X\). A function \(f: X \to Y\) is said to \textbf{respect the equivalence relation} if \(x_1 \sim x_2 \Rightarrow f(x_1) = f(x_2)\)  for \(x_1, x_2 \in X\). A set \(\overline{X}\)together with a function \(\pi: X \to \overline{X}\) is said to have the universal property of quotients if:
  \begin{enumerate}
  \item \(\pi: X \to \overline{X}\) respects the equivalence relation.
  \item For any function \(f: X \to Y\) such that \(f\) respects the equivalence relation, \(\exists! \overline{f}:\overline{X} \to Y\) such that

    \begin{center}
      \begin{tikzcd}
        X \arrow[r, "f"] \arrow[d, "\pi"] & Y \arrow[leftarrow, dl, "\overline{f}"]\\
        \overline{X} &
      \end{tikzcd}
    \end{center}
  \end{enumerate}
\end{df}

\begin{exa}
  in general, \(\pi: X \to X/\sim\) taking \(x \mapsto [x]\) has the universal property.
\end{exa}

\begin{exa}
  \(X = \Z\), \(a \sim b \Leftrightarrow a-b \equiv 0 \mod{6}\). Then \(X/\sim\) is called the integers modulo 6, written \(\Z/6\) (or more commonly \(\Z/6\Z\)).

  Let's define ``multiplying by 2''.

  \begin{center}
    \begin{tikzcd}
      \Z \arrow[r, "\times 2"] \arrow[d, "\pi"] & \Z/6 \\
      \Z/6 \arrow[dotted, ur] &
    \end{tikzcd}
  \end{center}

  well notice that what we really need here is
  \begin{center}
    \begin{tikzcd}
      a \arrow[mapsto, d] \arrow[mapsto, r] & \left[2a\right] \\
      \left[a\right] \arrow[mapsto, ur]
    \end{tikzcd}
  \end{center}

  So the thing that we really want is \(\Z \to \Z/6\) that takes \(a \mapsto [2a]\). Does this respect the equivalence relation?

  If \(a \sim b\), is it true that \([2a] = [2b]\)?

  \begin{align*}
    a \sim b &\Rightarrow a-b \equiv 0 \mod{6} \\
             &\Rightarrow a - b = 6n\ (n \in \Z) \\
             &\Rightarrow 2a -2b = 6(2n) \\
             &\Rightarrow 2a \sim 2b \\
             &\Rightarrow [2a] = [2b].
  \end{align*}
  So we have what we want.
  
\end{exa}


\begin{df}
  Let \(Z \subseteq \R^n\). A point \(\vec{x} \in \R^n\) is a \textbf{limit point of \(Z\)} if for all \(\epsilon > 0\), \[(B(\vec{x}, \epsilon) - \opn \vec{x} \cls) \cap Z \neq \emptyset.\]

  A point \(\vec{x} \in \R^n\) is said to be an \textbf{adherent point of \(Z\)} if for all \(\epsilon > 0\), \[B(\vec{x}, \epsilon) \cap Z \neq \emptyset.\] Equivalently, \(\vec{x}\) is adherent if \(\vec{x}\) is a limit point or \(\vec{x} \in Z\).
\end{df}


\begin{exa}
  Let
  \begin{align*}
    &f: \R \to \R \\
    &x \mapsto
    \begin{cases}
      0 \quad x < 0 \\
      20 \quad x \geq 0
    \end{cases}.
  \end{align*}

  Obviously, this should not be continuous. Recall the definition of continuity. Namely that \(\forall \epsilon > 0,\ \forall x\ \exists \delta > 0\ \forall y\ | x - y | < \delta \Rightarrow |f(x) - f(y)| < \epsilon\). Furthermore, non-continuity entails that \(\exists \epsilon > 0\ \exists x\ \forall \delta > 0\ \exists y\ |x - y| < \delta\) and \(|f(x) - f(y)| \geq \epsilon\).

  So let \(x = 0, \epsilon = 10.\) now let \(\delta > 0\). Let \(y = \frac{-\delta}{2}\). Then \(|x - y| = \frac{\delta}{2} < \delta\), but \(|f(x) - f(y)| = |20 -0| = 20 > 10 = \epsilon\)
\end{exa}

Now let's take \(\begin{matrix} g: \R \to \R \\  x \mapsto 3x\end{matrix}
\). So we can get \(\forall \epsilon > 0\ \forall x,\ \exists \delta > 0,\ \forall y\ |x - y| < \delta \Rightarrow |g(x) - g(y)| < \epsilon\).
Now let \(\epsilon > 0\), and \(x \in \R\). Take \(\delta = \frac{\epsilon}{3}\). Then for all \(y \in \R\) with \(|x -y| < \frac{\epsilon}{3}\),

\begin{align*}
  |g(x) - g(y)| &= |3x - 3y| \\
                &= 3|x - y| \\
                &< 3\left(\frac{\epsilon}{3}\right) \\
                &= \epsilon.
\end{align*}

So \(g\) is continuous.
\end{document}
%%% Local Variables:
%%% mode: latex
%%% TeX-master: t
%%% End:

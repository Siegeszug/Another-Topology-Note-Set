\documentclass[12pt, twosided]{article}

\usepackage[letterpaper,bindingoffset=0in,%
            left=1in,right=1in,top=1in,bottom=1in,%
            footskip=.25in]{geometry}

\usepackage{mathtools}
\usepackage{graphicx}

\usepackage{setspace}
\setstretch{1.1}

\usepackage{amsmath}
\usepackage{amsfonts}
\usepackage{amsthm}
\usepackage{amssymb}
\usepackage{csquotes}
\usepackage{relsize}

\usepackage{tikz}
\usetikzlibrary{cd}
\usetikzlibrary{fit,shapes.geometric}
\tikzset{%  
    mdot/.style={draw, circle, fill=black},
    mset/.style={draw, ellipse, very thick},
}

\usepackage{hhline}
\usepackage{systeme}
\usepackage{mathrsfs}
\usepackage{hyperref}
\usepackage{mathtools}  
\usepackage{silence}
\usepackage{blkarray}
\usepackage{float}
\usepackage{framed}
\usepackage{array}
\usepackage{stmaryrd}
\usepackage{extarrows}
\usepackage{caption}
\captionsetup[figure]{labelfont={bf},name={Fig.},labelsep=period}

\theoremstyle{definition}
\newtheorem{df}{Definition}
\newtheorem{exa}{Example}
\newtheorem{ques}{Question}
\newtheorem{exr}{Exercise}
\newtheorem*{note}{Note}
\theoremstyle{plain}
\newtheorem{thm}{Theorem}
\newtheorem{prop}{Proposition}
\newtheorem{conj}{Conjecture}
\newtheorem{cor}{Corollary}
\newtheorem{lm}{Lemma}
\newtheorem*{fact}{Fact}
\newtheorem*{idea}{Idea}
\newtheorem*{clm}{Claim}
\newtheorem*{rmk}{Remark}
\usepackage[ruled]{algorithm2e}

\usepackage{ulem}
\makeatletter

\def\lf{\left\lfloor}   
\def\rf{\right\rfloor}
\def\lc{\left\lceil}   
\def\rc{\right\rceil}
\def\st{\text{ s.t. }}
\def\1{^{-1}}
\def\ind{\mathbf{1}}
\def\R{\mathbb{R}}
\def\Q{\mathbb{Q}}
\def\Z{\mathbb{Z}}
\def\C{\mathbb{C}}
\def\I{\mathbb{I}}
\def\N{\mathbb{N}}
\def\F{\mathbb{F}}
\def\A{\mathbb{A}}
\def\Li{\text{Li}}
\def\th{^\text{th}}
\def\sp{\text{Sp}}
\def\opn{\left\{}
\def\cls{\right\}}
\def\Aut{\text{Aut}}
\def\PG{\text{PG}}
\def\GL{\text{GL}}
\def\PGL{\text{PGL}}
\def\Cov{\text{Cov}}
\def\Pack{\text{Pack}}
\def\PgamL{\text{P}\Gamma\text{L}}
\def\gamL{\Gamma\text{L}}
\def\cl{\text{cl}}
\def\stbar{\ \middle\vert\ }
\def\partdone{\hphantom{1} \hfill \(\triangle\)}
\def\s0{_0}
\def\s1{_1}
\def\s2{_2}
\def\id{\mathrm{id}}
\def\topn{\text{ open}}
\def\Bd{\text{Bd }}
\renewcommand{\P}{\mathbb{P}}
\newcommand{\leg}[2]{\left( \frac{#1}{#2} \right)}
\renewcommand*\env@matrix[1][*\c@MaxMatrixCols c]{%
   \hskip -\arraycolsep
   \let\@ifnextchar\new@ifnextchar
   \array{#1}}
\makeatother

% These two lines suppress the warning generated 
% by amsmath for overwriting the choose command  
% because it's annoying. This probably has unint-
% ended ramifications somewhere else, but I'm too
% lazy to actually figure that out, so we'll cro-
% ss that bridge when we come to it lol.
\renewcommand{\choose}[2]{\left( {#1 \atop #2} \right)}
\WarningFilter{amsmath}{Foreign command} 

\renewcommand{\mod}[1]{\ (\mathrm{mod}\ #1)}
\renewcommand{\vec}[1]{\mathbf{#1}}

\let\oldprime\prime
\def\prime{^\oldprime}

\usepackage{float}
\restylefloat{figure}

\usepackage{cleveref}
\Crefname{thm}{Theorem}{Theorems}

% Comment commands for co-authors
\newcommand{\kmd}[1]{{\color{purple} #1}}

\newcolumntype{L}{>{$}l<{$}}
% Bib matter
\let\oldepsilon\epsilon
\def\epsilon{\varepsilon}

\let\oldphi\phi
\def\phi{\varphi}

%%% Local Variables:
%%% mode: plain-tex
%%% TeX-master: t
%%% End:

\graphicspath{{./img/}}

\begin{document}
\noindent \textbf{Math 171} \hfill \textbf{Professor Sebastian Bozlee} \\
\textbf{Scribed by: Kyle Dituro} \hfill \textbf{March 31, 2023}\hrule
\vspace{.2in}

Let's cover a bit of the theorem that we stated during class yesterday before we get to the recitation.

\begin{thm}
  Let \(X, Y\) be topological spaces, let \(\opn U_i \cls_{i \in I}\) be an open cover of \(X\).
  \begin{enumerate}
  \item If \(f: X \to Y\), \(g: X \to Y\)then \(f = g\) iff \(f\vert_{U_i} = g\vert_{U_i}\) for all \(i \in I\).
  \item If \(f_i : U_i \to Y\) is a continuous function for each \(i \in I\) and \(f_i\vert_{U_i \cap U_j} = f_j\vert_{U_i \cap U_j}\) for all \(i, j \in I\) then there exists a unique continuous function \(f:X \to Y\) such that \(f\vert_{U_i} = f_i\) for all \(i \in I\).
  \end{enumerate}
\end{thm}
\begin{rmk}
  These two properties together are known as the ``sheaf property''.
\end{rmk}
Let's see if we can give this a proof

\begin{proof}
  \begin{enumerate}
  \item Suppose \(f = g\). Then clearly, \(f\vert_{U_i} = g\vert_{U_i}\) for all \(i \in I\).

    Conversely suppose \(f\vert_{U_i} = g\vert_{U_i}\) for all \(i \in I\). We want to show that \(f = g\). Let \(x \in X\). Since \(X = \bigcup_{i \in I} U_i\), there exists an \(i\) such that \(x \in U_i\). Then \(f(x) = f\vert_{U_i}(x) = g\vert_{U_i}(x) = g(x)\), and thus since \(X\) and \(x\) are arbitrary, \(f = g\). \partdone
  \item Let \(f: X \to Y\) be defined by \(f(x) = f_i(x)\) when \(x \in U_i\). Notice that there is a question of well-definedness here. To see if this \(f\) is well-defined, we need to check that \(f_i(x) = f_j(x)\) when \(x \in U_i\) and \(x \in U_j\) (for some \(i,j \in I\)). This follows directly from the second part of our assumption. Namely, \(x \in U_i \cap U_j\), and we know that \(f_i\vert_{U_i \cap U_j}(x) = f_j\vert_{U_i \cap U_j}(x)\), so \(f_i(x) = f_j(x)\) so \(f\) is well-defined.

    Note \(f\vert_{U_i} = f_i\) for each \(i \in I\) by definition. By the local characterization of continuity, we know that an everywhere locally continuous function is continuous, so we get that \(f\) is continuous.

    Now we need uniqueness. Suppose \(g: X \to Y\) is a second continuous function such that \(g\vert_{U_i} = f_i\) for all \(i \in I\). Then, by part (1), we get that \(f = g\)
  \end{enumerate}
\end{proof}

We better get to the actual recitation now...
\end{document}
%%% Local Variables:
%%% mode: latex
%%% TeX-master: t
%%% End:

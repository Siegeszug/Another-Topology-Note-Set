\documentclass[12pt, twosided]{article}

\usepackage[letterpaper,bindingoffset=0in,%
            left=1in,right=1in,top=1in,bottom=1in,%
            footskip=.25in]{geometry}

\usepackage{mathtools}
\usepackage{graphicx}

\usepackage{setspace}
\setstretch{1.1}

\usepackage{amsmath}
\usepackage{amsfonts}
\usepackage{amsthm}
\usepackage{amssymb}
\usepackage{csquotes}
\usepackage{relsize}

\usepackage{tikz}
\usetikzlibrary{cd}
\usetikzlibrary{fit,shapes.geometric}
\tikzset{%  
    mdot/.style={draw, circle, fill=black},
    mset/.style={draw, ellipse, very thick},
}

\usepackage{hhline}
\usepackage{systeme}
\usepackage{mathrsfs}
\usepackage{hyperref}
\usepackage{mathtools}  
\usepackage{silence}
\usepackage{blkarray}
\usepackage{float}
\usepackage{framed}
\usepackage{array}
\usepackage{stmaryrd}
\usepackage{extarrows}
\usepackage{caption}
\captionsetup[figure]{labelfont={bf},name={Fig.},labelsep=period}

\theoremstyle{definition}
\newtheorem{df}{Definition}
\newtheorem{exa}{Example}
\newtheorem{ques}{Question}
\newtheorem{exr}{Exercise}
\newtheorem*{note}{Note}
\theoremstyle{plain}
\newtheorem{thm}{Theorem}
\newtheorem{prop}{Proposition}
\newtheorem{conj}{Conjecture}
\newtheorem{cor}{Corollary}
\newtheorem{lm}{Lemma}
\newtheorem*{fact}{Fact}
\newtheorem*{idea}{Idea}
\newtheorem*{clm}{Claim}
\newtheorem*{rmk}{Remark}
\usepackage[ruled]{algorithm2e}

\usepackage{ulem}
\makeatletter

\def\lf{\left\lfloor}   
\def\rf{\right\rfloor}
\def\lc{\left\lceil}   
\def\rc{\right\rceil}
\def\st{\text{ s.t. }}
\def\1{^{-1}}
\def\ind{\mathbf{1}}
\def\R{\mathbb{R}}
\def\Q{\mathbb{Q}}
\def\Z{\mathbb{Z}}
\def\C{\mathbb{C}}
\def\I{\mathbb{I}}
\def\N{\mathbb{N}}
\def\F{\mathbb{F}}
\def\A{\mathbb{A}}
\def\Li{\text{Li}}
\def\th{^\text{th}}
\def\sp{\text{Sp}}
\def\opn{\left\{}
\def\cls{\right\}}
\def\Aut{\text{Aut}}
\def\PG{\text{PG}}
\def\GL{\text{GL}}
\def\PGL{\text{PGL}}
\def\Cov{\text{Cov}}
\def\Pack{\text{Pack}}
\def\PgamL{\text{P}\Gamma\text{L}}
\def\gamL{\Gamma\text{L}}
\def\cl{\text{cl}}
\def\stbar{\ \middle\vert\ }
\def\partdone{\hphantom{1} \hfill \(\triangle\)}
\def\s0{_0}
\def\s1{_1}
\def\s2{_2}
\def\id{\mathrm{id}}
\def\topn{\text{ open}}
\def\Bd{\text{Bd }}
\renewcommand{\P}{\mathbb{P}}
\newcommand{\leg}[2]{\left( \frac{#1}{#2} \right)}
\renewcommand*\env@matrix[1][*\c@MaxMatrixCols c]{%
   \hskip -\arraycolsep
   \let\@ifnextchar\new@ifnextchar
   \array{#1}}
\makeatother

% These two lines suppress the warning generated 
% by amsmath for overwriting the choose command  
% because it's annoying. This probably has unint-
% ended ramifications somewhere else, but I'm too
% lazy to actually figure that out, so we'll cro-
% ss that bridge when we come to it lol.
\renewcommand{\choose}[2]{\left( {#1 \atop #2} \right)}
\WarningFilter{amsmath}{Foreign command} 

\renewcommand{\mod}[1]{\ (\mathrm{mod}\ #1)}
\renewcommand{\vec}[1]{\mathbf{#1}}

\let\oldprime\prime
\def\prime{^\oldprime}

\usepackage{float}
\restylefloat{figure}

\usepackage{cleveref}
\Crefname{thm}{Theorem}{Theorems}

% Comment commands for co-authors
\newcommand{\kmd}[1]{{\color{purple} #1}}

\newcolumntype{L}{>{$}l<{$}}
% Bib matter
\let\oldepsilon\epsilon
\def\epsilon{\varepsilon}

\let\oldphi\phi
\def\phi{\varphi}

%%% Local Variables:
%%% mode: plain-tex
%%% TeX-master: t
%%% End:

\graphicspath{{./img/}}

\begin{document}
\noindent \textbf{Math 171} \hfill \textbf{Professor Sebastian Bozlee} \\
\textbf{Scribed by: Kyle Dituro} \hfill \textbf{March 7, 2023}\hrule
\vspace{.2in}

We start with a final theorem on Bases

\begin{lm}
  Let \(X\) be a topological space. Suppose \(\mathcal{C}\) is a collection of open subsets of \(X\) such that for each open subset \(U\) of \(X\) and each \(x \in U\), there is an element \(C \in \mathcal{C}\) such that \(x \in C \subseteq U\). Then \(C\) is a basis for the topology on \(X\).
\end{lm}
\begin{proof}
  We verify the basis axioms first:
  \begin{enumerate}
  \item \textbf{If \(x \in X\), we want to show that then \(\exists C \in \mathcal{C}\) such that \(x \in C\).}
    
    By hypothesis: since \(U = X\) is an open subset of \(X\), for any \(x \in X\), there is a \(C \in \mathcal{C}\) such that \(x \in C\), and the first axiom is verified.
    
  \item \textbf{If \(C_1, C_2 \in \mathcal{C}\) and \(x \in C_1 \cap C_2\), then \(\exists C_3 \in \mathcal{C}\) such that \(x \in C_3 \subseteq C_1 \cap C_2\).}

    Let \(C_1, C_2 \in \mathcal{C}\) and \(x \in C_1 \cap C_2\). Then by definition, \(C_1, C_2\) are open, so \(C_1 \cap C_2\) is open, and then by assumption, for \(U = C_1 \cap C_2\), there exists \(C_3 \in C\) such that \(x \in C_3 \subseteq C_1 \cap C_2\)
  \end{enumerate} \partdone

  So now we want to verify that the topology on \(X\) is that generated by \(\mathcal{C}\), which we will do by double-containment on the topologies.

  \begin{enumerate}
  \item [(\(\Rightarrow\))] Suppose that \(U\) is open in \(X\). For each \(x \in U\), there is an element \(C_x\) of \(\mathcal{C}\) such that \(x \in C_x \subseteq U\). Then
    \begin{align*}
      U &= \bigcup_{x \in U} C_x,
    \end{align*}
    So \(U\) is open in the topology generated by \(C\).
  \item [\((\Leftarrow)\)] Now suppose that \(U \subseteq X\) that is open in the generated topology. Then \(U = \bigcup_{i \in I} C_i\) where \(C_i \in \mathcal{C}\) for each \(i \in I\). This is a union of open subsets of \(X\), so \(U\) is open in the original topology on \(X\).
  \end{enumerate}
  And thus we have double containment, and the lemma is proven.
\end{proof}

Now we will begin our discussion of homeomorphisms.

When we are working in the category of sets, the `tool of choice' is functions between sets. Namely, an invertible function tells us a way in which two sets can be considered equivalent.

Likewise, we have found that in the category of topological spaces, continuous functions are our `tool of choice'. Likewise, invertible continuous functions will be our form of equivalence of topological spaces.

\begin{df}
  Let \(X, Y\) be topological spaces. A \textbf{Homeomorphism from \(X \to Y\)} is a continuous function \(f: X \to Y\) with inverse function \(f\1:Y \to X\).
\end{df}

\begin{exa}
  Notice that a continuous bijection need not always have a continuous inverse. Consider the identity function. \(\mathrm{id}_{\{1, 2\}}:(\opn 1, 2 \cls, \tau_{\mathrm{disc}}) \to (\opn 1, 2 \cls, \tau_{\mathrm{indisc}})\). As we have studied, any function from the discrete topology is continuous, and any function from the indiscreet topology is not, so obviously our function here is continuous without a continuous inverse.
\end{exa}

Following this non-example, we give an actual example.

\begin{exa}
  Let \(X = [0, 1], Y = [0, 2]\). Then take \(
  \begin{matrix}
    f: X \to Y \\ x \mapsto 2x 
  \end{matrix} \), which is continuous and bijective, with inverse \(
  \begin{matrix}
    f\1: Y \to X \\ y \mapsto \frac{y}{2}
  \end{matrix}\). This describes what we mean by the fact that topology has no sense of ``distance''.
\end{exa}

\begin{df}
  If there exists a homeomorphism \(f: X \to Y\), we say \(X\) is \textbf{homeomorphic} to \(Y\), and we write \(X \cong Y\).
\end{df}

\begin{prop}
  Let \(f: X \to Y\) be a homeomorphism, \(g: Y \to X\) its continuous inverse. Then:
  \begin{enumerate}
  \item \[
      \begin{matrix}
        \opn U \in X \stbar U \text{ is open} \cls & \longleftrightarrow & \opn V \subseteq Y \stbar V \text{ is open}\cls \\
        U & \longmapsto & g\1(U) \\
        f\1(v) & \longmapsfrom & V
      \end{matrix}
    \]
  \item \[
      \begin{matrix}
        \opn W \subseteq X \stbar W \text{ is closed} \cls & \longleftrightarrow & \opn Z \subseteq Y \stbar Z \text{ is closed} \cls \\
        W & \longmapsto & g\1(W) \\
        f\1(Z) & \longmapsfrom & Z
      \end{matrix}
    \]
  \end{enumerate}
\end{prop}

\begin{proof}
  \begin{enumerate}
  \item The two maps are well-defined because \(f\) and \(g\) are continuous. To see that the maps are inverses, note that \(\forall U \subseteq X\) open, we have
    \begin{align*}
      f\1(g\1(U)) &= (g \circ f)\1 \\
                  &= \mathrm{id}_X(U) \\
                  &= U.
    \end{align*}
    And we do similarly for all other cases.
  \end{enumerate}
\end{proof}

\begin{exa}
  \((0, 1) \cong \R\), with \(f: (0, 1) \to \R\) via \(f(x) = \tan\left(\pi\left(x - \frac{1}{2}\right)\right)\)
\end{exa}

\begin{lm}
  Rational functions, root functions, exponentials, logs, trig functions, and inverse trig functions are all continuous on their domains.
\end{lm}

\begin{note}
  A rational function is a quotient of polynomial functions:
  \begin{align*}
    f(x_1, \ldots, x_n) = \frac{P(x_1, \ldots, x_n)}{Q(x_1 \ldots, x_n)}
  \end{align*}
\end{note}

\begin{lm}
  Let \(X, Y, Z\) be topological spaces. Then
  \begin{enumerate}
  \item \(\mathrm{id}_X: X \to X\) is a homeomorphism
  \item If \(f: X \to Y\)is a homeomorphism, so is \(f\1: Y \to X\).
  \item If \(f: X \to Y\), \(g: Y \to Z\), then \(g \circ f : X \to Z\) is also a homeomorphism
  \end{enumerate}
\end{lm}
\begin{proof}
  \begin{enumerate}
  \item Obvious (what's the inverse of \(\mathrm{id}_X\)?)\partdone
  \item Since \(f\) is a homeomorphism:
    \begin{itemize}
    \item \(f\) is continuous
    \item \(f\) has an inverse \(f\1\)
    \item \(f\1\) is continuous/
    \end{itemize}
    So...
    \begin{itemize}
    \item \(f\1\) is continuous
    \item \(f\1\) has an inverse \(f\)
    \item and \(f\1\) is a homeomorphism
    \end{itemize}\partdone
  \item \(f: X \to Y\) and \(g: Y \to Z\) are homeomorphisms, so \(f, g\) are continuous and have continuous inverses \(f\1, g\1\). So:
    \begin{itemize}
    \item \(g \circ f\) is continuous as a composite of continuous functions
    \item \((g \circ f)\1\) is an inverse.
    \item \(f\1 \circ g\1\) is continuous.
    \end{itemize}
  \end{enumerate}
\end{proof}

\begin{lm}
  Let \(X, Y, Z\) be topological spaces. Then:
  \begin{enumerate}
  \item \(X \cong X\)
  \item If \(X \cong Y\), then \(Y \cong X\),
  \item If \(X \cong Y\), and \(Y \cong Z\), then \(X \cong Z\).
  \end{enumerate}
\end{lm}
\begin{proof}
  Admitted.
  \textbf{Hint:} Apply the definition of homeomorphic to the previous lemma.
\end{proof}
\end{document}
%%% Local Variables:
%%% mode: latex
%%% TeX-master: t
%%% End:

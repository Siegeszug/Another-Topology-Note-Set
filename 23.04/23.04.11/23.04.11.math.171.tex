\documentclass[12pt, twosided]{article}

\usepackage[letterpaper,bindingoffset=0in,%
            left=1in,right=1in,top=1in,bottom=1in,%
            footskip=.25in]{geometry}

\usepackage{mathtools}
\usepackage{graphicx}

\usepackage{setspace}
\setstretch{1.1}

\usepackage{amsmath}
\usepackage{amsfonts}
\usepackage{amsthm}
\usepackage{amssymb}
\usepackage{csquotes}
\usepackage{relsize}

\usepackage{tikz}
\usetikzlibrary{cd}
\usetikzlibrary{fit,shapes.geometric}
\tikzset{%  
    mdot/.style={draw, circle, fill=black},
    mset/.style={draw, ellipse, very thick},
}

\usepackage{hhline}
\usepackage{systeme}
\usepackage{mathrsfs}
\usepackage{hyperref}
\usepackage{mathtools}  
\usepackage{silence}
\usepackage{blkarray}
\usepackage{float}
\usepackage{framed}
\usepackage{array}
\usepackage{stmaryrd}
\usepackage{extarrows}
\usepackage{caption}
\captionsetup[figure]{labelfont={bf},name={Fig.},labelsep=period}

\theoremstyle{definition}
\newtheorem{df}{Definition}
\newtheorem{exa}{Example}
\newtheorem{ques}{Question}
\newtheorem{exr}{Exercise}
\newtheorem*{note}{Note}
\theoremstyle{plain}
\newtheorem{thm}{Theorem}
\newtheorem{prop}{Proposition}
\newtheorem{conj}{Conjecture}
\newtheorem{cor}{Corollary}
\newtheorem{lm}{Lemma}
\newtheorem*{fact}{Fact}
\newtheorem*{idea}{Idea}
\newtheorem*{clm}{Claim}
\newtheorem*{rmk}{Remark}
\usepackage[ruled]{algorithm2e}

\usepackage{ulem}
\makeatletter

\def\lf{\left\lfloor}   
\def\rf{\right\rfloor}
\def\lc{\left\lceil}   
\def\rc{\right\rceil}
\def\st{\text{ s.t. }}
\def\1{^{-1}}
\def\ind{\mathbf{1}}
\def\R{\mathbb{R}}
\def\Q{\mathbb{Q}}
\def\Z{\mathbb{Z}}
\def\C{\mathbb{C}}
\def\I{\mathbb{I}}
\def\N{\mathbb{N}}
\def\F{\mathbb{F}}
\def\A{\mathbb{A}}
\def\Li{\text{Li}}
\def\th{^\text{th}}
\def\sp{\text{Sp}}
\def\opn{\left\{}
\def\cls{\right\}}
\def\Aut{\text{Aut}}
\def\PG{\text{PG}}
\def\GL{\text{GL}}
\def\PGL{\text{PGL}}
\def\Cov{\text{Cov}}
\def\Pack{\text{Pack}}
\def\PgamL{\text{P}\Gamma\text{L}}
\def\gamL{\Gamma\text{L}}
\def\cl{\text{cl}}
\def\stbar{\ \middle\vert\ }
\def\partdone{\hphantom{1} \hfill \(\triangle\)}
\def\s0{_0}
\def\s1{_1}
\def\s2{_2}
\def\id{\mathrm{id}}
\def\topn{\text{ open}}
\def\Bd{\text{Bd }}
\renewcommand{\P}{\mathbb{P}}
\newcommand{\leg}[2]{\left( \frac{#1}{#2} \right)}
\renewcommand*\env@matrix[1][*\c@MaxMatrixCols c]{%
   \hskip -\arraycolsep
   \let\@ifnextchar\new@ifnextchar
   \array{#1}}
\makeatother

% These two lines suppress the warning generated 
% by amsmath for overwriting the choose command  
% because it's annoying. This probably has unint-
% ended ramifications somewhere else, but I'm too
% lazy to actually figure that out, so we'll cro-
% ss that bridge when we come to it lol.
\renewcommand{\choose}[2]{\left( {#1 \atop #2} \right)}
\WarningFilter{amsmath}{Foreign command} 

\renewcommand{\mod}[1]{\ (\mathrm{mod}\ #1)}
\renewcommand{\vec}[1]{\mathbf{#1}}

\let\oldprime\prime
\def\prime{^\oldprime}

\usepackage{float}
\restylefloat{figure}

\usepackage{cleveref}
\Crefname{thm}{Theorem}{Theorems}

% Comment commands for co-authors
\newcommand{\kmd}[1]{{\color{purple} #1}}

\newcolumntype{L}{>{$}l<{$}}
% Bib matter
\let\oldepsilon\epsilon
\def\epsilon{\varepsilon}

\let\oldphi\phi
\def\phi{\varphi}

%%% Local Variables:
%%% mode: plain-tex
%%% TeX-master: t
%%% End:

\graphicspath{{./img/}}

\begin{document}
\noindent \textbf{Math 171} \hfill \textbf{Professor Sebastian Bozlee} \\
\textbf{Scribed by: Kyle Dituro} \hfill \textbf{April 11, 2023}\hrule
\vspace{.2in}

\begin{cor}[``Images are almost quotient spaces'']
  Let \(g: X \to Z\) be a surjective continuous map.
  Let \(\sim\) be the equivalence relation on \(X\) given by

  \begin{align*}
    x_1 \sim x_2 \Longleftrightarrow g(x_1) = g(x_2).
  \end{align*}

  Then \(g\) induces a continuous \textbf{bijective} map \(f: X/\sim\ \to Z\) given by \(f([x]) = g(x)\), which is a homeomorphism if and only if \(g\) is a quotient.

  \begin{center}
    \begin{tikzcd}
      X \arrow[dr, "g"]& \\
      X/\sim \arrow[u, leftarrow, "\pi"] & Z  \arrow[l, leftarrow, "f"]
    \end{tikzcd}
  \end{center}
\end{cor}

\begin{proof}
  It is clear that \(\sim\) is an equivalence relation.

  We know \(f\) is well defined and continuous from the universal property of the quotient.

  It is... uhhh... ``clear'' that \(f\) is bijective:
  \begin{itemize}
  \item (\textbf{injective}): Pretty clear
  \item (\textbf{surjective}): Let \(z \in Z\). Since \(g\) is surjective, \(\exists x \in X\) such that \(g(x) = z\). Then \(f([x]) = z\).
  \end{itemize}

  Suppose \(f\) is a homeomorphism. Then in particular \(f\) is a quotient map:

  \begin{align*}
    U \subseteq Z \topn \Longleftrightarrow f\1(U) \subseteq X/\sim\topn
  \end{align*}

  Then \(g = f \circ \pi\) is a composite of quotient maps, and is therefore a quotient map.

  Conversely, if \(g\) is a quotient map, then it also has the universal property of the quotient:

  By the universal property of \(g\), \(\exists ! h: Z \to X/\sim\) such that

  \begin{center}
    \begin{tikzcd}
      X \arrow[dr, "g"]& \\
      X/\sim \arrow[u, leftarrow, "\pi"] & Z \arrow[l, "h"] 
    \end{tikzcd}
  \end{center}

  Taking composites:

    \begin{center}
    \begin{tikzcd}
      & X \arrow[dr, "\pi"]& \\
      X/\sim \arrow[ur, leftarrow, "\pi"]& & X/\sim \arrow[ll, leftarrow, "h \circ f"] 
    \end{tikzcd}
  \end{center}
  So by the uniqueness in the universal property for \(\pi\), we get that \(h \circ f = \id_{X/\sim}\).

  Symmetrically, \(f \circ h = \id_Z\), s \(f\) is a homeomorphism, as desired.
\end{proof}

\begin{exa}
  Recall \(\R\P^2 = \R^3 - \opn 0 \cls / \sim\).

  This has a particularly nice open subset \(D(z)\):

  \begin{align*}
    D(z) = \opn [x : y : z] \in \R\P^2 \stbar z \neq 0 \cls.
  \end{align*}

  This is homeomorphic to \(\R^2\) via

  \begin{align*}
    \R^2 &\to D(z) \\
    (x, y) &\mapsto [x: y : 1] \\
    \left(\frac{x}{z}, \frac{y}{z}\right) &\mapsfrom [x: y: z]
  \end{align*}

  Similarly, there are open subsets

  \begin{align*}
    D(x) = \opn [x: y: z] \stbar x \neq 0 \cls \cong \R^2 \\
    D(y) = \opn [x: y: z] \stbar y \neq 0 \cls \cong \R^2 \\
  \end{align*}

  Notice that these open sets form an open cover of \(\R\P^2\): \(D(x) \cup D(y) \cup D(z) = \R\P^2\).

  This is nice.
\end{exa}

\begin{df}
  A topological space \(X\) is said to be a \textbf{Topological Manifold} if it has an open cover \(\opn U_i \cls_{i \in I}\) such that:
  \begin{enumerate}
  \item each \(U_i\) is homeomorphic to an open subset of \(\R^n\) for some \(n\) (\(n\) the dimension of \(X\))
  \item \(X\) is Hausdorff 
  \item \sout{something technical that we don't care about}
  \end{enumerate}
\end{df}

\begin{exa}
  \(\R^n, \R\P^n, \mathbb{S}^n = \opn \vec{x} \in \R^{n+1} \stbar |\vec{x} | = 1 \cls\), donuts.
\end{exa}


\begin{exa}
  \begin{align*}
    \R\P^1 &= \opn (x, y) \stbar (x, y) \neq 0 \cls / \sim \text{ scaling} \\
    D(x) &= \opn [x, y] \stbar x \neq 0 \cls \xrightarrow{\sim} \R^1 \\
           & [x:y] \longmapsto \frac{y}{x} \\
    D(y) &= \opn [x:y] \stbar y \neq 0 \cls \xrightarrow{\sim} \R^1 \\
           & [x:y] \longmapsto \frac{x}{y}.
  \end{align*}
\end{exa}

we now give the gluing construction:

\begin{itemize}
\item [Input:] Two top spaces \(U_1\), \(U_2\) \\
  An open subset \(U_{12}\) of \(U_1\)\\
  An open subset \(U_{21}\) of \(U_2\)\\
  A homeomorphism \(\phi_{12}: U_{12} \to U_{21}\)
\item [Output:] \(X = U_1 \sqcup U_2 / \sim\ = \opn 1 \cls \times U_1 \cup \opn 2 \cls \times U_2 / \sim \) Where \((1, u) \sim (2, \phi_{12}(u))\) when \(u \in U_{12}\).
\end{itemize}

\begin{thm}
  X has an open cover \(V_1, V_2\) such that there exists homeomorphisms

  \begin{enumerate}
  \item
    \(\begin{matrix}
      \phi_1: U_1 \to V_1 \\
      \phi_2: U_2 \to V_2
    \end{matrix}\)
  \item\(\phi_1(U_{12}) = \phi_2(U_{21})\)
  \item \(\phi\1_2 \circ \phi_1 : U_{12} \to U_{21}\) is equal to \(\phi_{12}\)
  \end{enumerate}
\end{thm}

\begin{exa}
  \begin{align*}
    \begin{matrix}
      U_1 = \R & U_{12} = \R - \opn 0 \cls \\
      U_2 = \R & U_{21} = \R - \opn 0 \cls
    \end{matrix}
  \end{align*}

  \begin{align*}
    \begin{matrix}
      \phi:&U_{12} \to U_{21} \\
           &x \mapsto \frac{1}{x}
    \end{matrix}
  \end{align*}

  This glues to \(\R\P^1\) 
\end{exa}

\begin{exa}
  Now take the same \(U\)'s but instead say \(x \xmapsto{\phi} x\). Then you get a cylinder
\end{exa}
\end{document}
%%% Local Variables:
%%% mode: latex
%%% TeX-master: t
%%% End:

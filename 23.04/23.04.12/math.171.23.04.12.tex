\documentclass[12pt, twosided]{article}

\usepackage[letterpaper,bindingoffset=0in,%
            left=1in,right=1in,top=1in,bottom=1in,%
            footskip=.25in]{geometry}

\usepackage{mathtools}
\usepackage{graphicx}

\usepackage{setspace}
\setstretch{1.1}

\usepackage{amsmath}
\usepackage{amsfonts}
\usepackage{amsthm}
\usepackage{amssymb}
\usepackage{csquotes}
\usepackage{relsize}

\usepackage{tikz}
\usetikzlibrary{cd}
\usetikzlibrary{fit,shapes.geometric}
\tikzset{%  
    mdot/.style={draw, circle, fill=black},
    mset/.style={draw, ellipse, very thick},
}

\usepackage{hhline}
\usepackage{systeme}
\usepackage{mathrsfs}
\usepackage{hyperref}
\usepackage{mathtools}  
\usepackage{silence}
\usepackage{blkarray}
\usepackage{float}
\usepackage{framed}
\usepackage{array}
\usepackage{stmaryrd}
\usepackage{extarrows}
\usepackage{caption}
\captionsetup[figure]{labelfont={bf},name={Fig.},labelsep=period}

\theoremstyle{definition}
\newtheorem{df}{Definition}
\newtheorem{exa}{Example}
\newtheorem{ques}{Question}
\newtheorem{exr}{Exercise}
\newtheorem*{note}{Note}
\theoremstyle{plain}
\newtheorem{thm}{Theorem}
\newtheorem{prop}{Proposition}
\newtheorem{conj}{Conjecture}
\newtheorem{cor}{Corollary}
\newtheorem{lm}{Lemma}
\newtheorem*{fact}{Fact}
\newtheorem*{idea}{Idea}
\newtheorem*{clm}{Claim}
\newtheorem*{rmk}{Remark}
\usepackage[ruled]{algorithm2e}

\usepackage{ulem}
\makeatletter

\def\lf{\left\lfloor}   
\def\rf{\right\rfloor}
\def\lc{\left\lceil}   
\def\rc{\right\rceil}
\def\st{\text{ s.t. }}
\def\1{^{-1}}
\def\ind{\mathbf{1}}
\def\R{\mathbb{R}}
\def\Q{\mathbb{Q}}
\def\Z{\mathbb{Z}}
\def\C{\mathbb{C}}
\def\I{\mathbb{I}}
\def\N{\mathbb{N}}
\def\F{\mathbb{F}}
\def\A{\mathbb{A}}
\def\Li{\text{Li}}
\def\th{^\text{th}}
\def\sp{\text{Sp}}
\def\opn{\left\{}
\def\cls{\right\}}
\def\Aut{\text{Aut}}
\def\PG{\text{PG}}
\def\GL{\text{GL}}
\def\PGL{\text{PGL}}
\def\Cov{\text{Cov}}
\def\Pack{\text{Pack}}
\def\PgamL{\text{P}\Gamma\text{L}}
\def\gamL{\Gamma\text{L}}
\def\cl{\text{cl}}
\def\stbar{\ \middle\vert\ }
\def\partdone{\hphantom{1} \hfill \(\triangle\)}
\def\s0{_0}
\def\s1{_1}
\def\s2{_2}
\def\id{\mathrm{id}}
\def\topn{\text{ open}}
\def\Bd{\text{Bd }}
\renewcommand{\P}{\mathbb{P}}
\newcommand{\leg}[2]{\left( \frac{#1}{#2} \right)}
\renewcommand*\env@matrix[1][*\c@MaxMatrixCols c]{%
   \hskip -\arraycolsep
   \let\@ifnextchar\new@ifnextchar
   \array{#1}}
\makeatother

% These two lines suppress the warning generated 
% by amsmath for overwriting the choose command  
% because it's annoying. This probably has unint-
% ended ramifications somewhere else, but I'm too
% lazy to actually figure that out, so we'll cro-
% ss that bridge when we come to it lol.
\renewcommand{\choose}[2]{\left( {#1 \atop #2} \right)}
\WarningFilter{amsmath}{Foreign command} 

\renewcommand{\mod}[1]{\ (\mathrm{mod}\ #1)}
\renewcommand{\vec}[1]{\mathbf{#1}}

\let\oldprime\prime
\def\prime{^\oldprime}

\usepackage{float}
\restylefloat{figure}

\usepackage{cleveref}
\Crefname{thm}{Theorem}{Theorems}

% Comment commands for co-authors
\newcommand{\kmd}[1]{{\color{purple} #1}}

\newcolumntype{L}{>{$}l<{$}}
% Bib matter
\let\oldepsilon\epsilon
\def\epsilon{\varepsilon}

\let\oldphi\phi
\def\phi{\varphi}

%%% Local Variables:
%%% mode: plain-tex
%%% TeX-master: t
%%% End:

\graphicspath{{./img/}}

\begin{document}
\noindent \textbf{Math 171} \hfill \textbf{Professor Sebastian Bozlee} \\
\textbf{Scribed by: Kyle Dituro} \hfill \textbf{\today}\hrule
\vspace{.2in}

We begin with a quick reminder that, along \(\mathbb{S}^1\), the open sets are open sets in \(\R^2\) which are intersected with the circle, and likewise that half-open intervals at the endpoints of \([0, 1]\) are also open.

Now moving on to connectedness.


\begin{df}
  Let \(X\) be a topological space. A \textbf{separation} of \(X\) is a pair \(U, V\) of disjoint, nonempty, open subsets of \(X\) whose union is all of \(X\). We say that \(X\) is \textbf{connected} if \(X\) does not have a separation.
\end{df}


\begin{rmk}
  \begin{itemize}
  \item The definition of connected is homeomorphism invariant
  \item If \(U, V\) is a separation of \(X\), then \(U^c = V\) and \(V^c = U\) are open, so \(U\) and \(V\) are both open and closed.
  \item \(X\) is connected iff the only subsets of \(X\) that are both open and closed are \(\emptyset\) and \(X\) itself. (To see this, simply take a complement of a nontrivial clopen set. What do you get?)
  \end{itemize}
\end{rmk}

\begin{exa}
  \begin{itemize}
  \item Consider \(X = \opn 1, 2 \cls\) with the indiscrete topology. This is clearly connected.
  \item Let \(X = [-1, 0) \cup (0, 1]\) w/ subspace topology. This is disconnected, since both \([-1, 0)\) and \((0,1]\) are disjoint open sets\footnote{Remember, this is in the subspace topology on \(\R\)} which union to the entire space.
  \item Let \(X = [-1, 1]\). This is connected, but that's kind of a truck to prove.
  \item Let \(X = \Q\). The only subspaces of \(\Q\) which are connected are the singleton sets. If \(Y\) is any subspace of \(\Q\) containing \(p < q\), then there is an irrational number \(a\) between \(p\) and \(q\). Then \(U = (-\infty, a) \cap Y\), \(V = (a, \infty) \cap Y\) is a separation of \(Y\).
  \end{itemize}
\end{exa}

\begin{thm}
  Let \(X\) be a connected topological space, \(f: X \to Y\) a continuous function. Then \(f(X)\) is connected (w.r.t. the subspace topology)
\end{thm}
\begin{proof}
  Let \(Z = f(X)\), and notice that \(
  \begin{matrix}
    g: X \to Z \\ x \mapsto f(x)
  \end{matrix}
  \) is also continuous.

  So suppose for the sake of contradiction that there exists a separation \(U, V\) of \(Z\). Then we claim that \(U\prime = g\1(U)\), \(V\prime = g\1(V)\) is a separation of \(X\).
  \begin{itemize}
  \item \textit{(disjoint):} \(U\prime \cap V\prime = g\1(U) \cap g\1(V) = g\1(U \cap V) = g\1(\emptyset) = \emptyset\)
  \item \textit{(nonempty):} \(U\prime, V\prime \neq \emptyset\) since \(g\) is surjective.
  \item \textit{(open):} Follows trivially since \(g\) is continuous.
  \item \textit{(Union is \(X\)):} \(U\prime \cup V\prime = g\1(U) \cup g\1(V) = g\1(U \cup V) = g\1(Z) = X\).
  \end{itemize}

  But then we've given a separation of \(X\), which is connected. A contradiction! So we must have that \(f(x) = Z\) is connected.
\end{proof}

Now we'll prove a sequence of lemmas, with the goal of proving that product spaces are connected.

\begin{lm}
  Suppose that \(X\) is a topological space, and further suppose that \(A\) is a connected subspace. If \(U, V\) is a separation of \(X\), then \(A \subseteq U\) or \(A \subseteq V\).
\end{lm}
\begin{proof}
  Consider \(U\prime = U \cap A\) and \(V\prime = V \cap A\). Then \(U\prime\) and \(V\prime\) are disjoint, open in \(A\) and \(U\prime \cup V\prime = A\).

  Since \(A\) is connected, it must be so that \(U\prime\) or \(V\prime\) is empty, otherwise we would have a separation.

  If \(U\prime = \emptyset\), \(A \subseteq V\), and if \(V\prime = \emptyset\), then \(A \subseteq U\).
\end{proof}

\begin{thm}
  If \(\opn A_i \cls_{i \in I}\) is a collection of connected subspaces of a topological space \(X\) and \(\bigcap_{i \in I} A_i \neq \emptyset\), then \(\bigcup_{i \in I} A_i\) (with the subspace topology) is connected.
\end{thm}
\begin{proof}
  Suppose -- again for the sake of contradiction -- that \(U, V\) is a separation of \(\bigcup_{i \in I} A_i\).

  Let \(p \in \bigcup_{i \in I} A_i\). Suppose without loss of generality that \(p \in U\).

  Consider \(A_i\) for some \(i \in I\). Then by the lemma, \(A_i \subseteq U\) or \(A_i \subseteq V\). Since \(p \in A_i\) and \(p \in U\), it must follow that \(A_i \subseteq U\). Since this is true for all \(i\), \(\bigcup_{i \in I} A_i \subseteq U\). But then \(V = \bigcup_{i \in I} A_i - U = \emptyset\), again a contradiction.

  So \(\bigcup_{i \in I} A_i\) must be connected.
\end{proof}

\begin{thm}
  Let \(X, Y\) be connected topological spaces. Then \(X \times Y\) is connected as well.
\end{thm}

\begin{proof}
  First, notice that if \(X, Y = \emptyset\), then \(X \times Y = \emptyset\) is connected.

  So let's suppose that \(X\) and \(Y\) are not empty. Then \(\exists (a, b) \in X \times Y\).

  Notice: \(X \times \opn b \cls \cong X\) is connected. Likewise, for any \(x \in X\), \(\opn x \cls \times Y \cong Y\) is connected.

  Thus by the lemma, \(T_x = ( X \times \opn b \cls) \cup (\opn x \cls \times Y)\) is connected.

  Then \(\bigcup_{x \in X} T_x = X \times Y\) and \((a, b) \in \bigcup_{x \in X} T_x\). So again by the lemma, \(X \times Y\) is connected.
\end{proof}
\end{document}
%%% Local Variables:
%%% mode: latex
%%% TeX-master: t
%%% End:

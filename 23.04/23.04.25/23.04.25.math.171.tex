\documentclass[12pt, twosided]{article}

\usepackage[letterpaper,bindingoffset=0in,%
            left=1in,right=1in,top=1in,bottom=1in,%
            footskip=.25in]{geometry}

\usepackage{mathtools}
\usepackage{graphicx}

\usepackage{setspace}
\setstretch{1.1}

\usepackage{amsmath}
\usepackage{amsfonts}
\usepackage{amsthm}
\usepackage{amssymb}
\usepackage{csquotes}
\usepackage{relsize}

\usepackage{tikz}
\usetikzlibrary{cd}
\usetikzlibrary{fit,shapes.geometric}
\tikzset{%  
    mdot/.style={draw, circle, fill=black},
    mset/.style={draw, ellipse, very thick},
}

\usepackage{hhline}
\usepackage{systeme}
\usepackage{mathrsfs}
\usepackage{hyperref}
\usepackage{mathtools}  
\usepackage{silence}
\usepackage{blkarray}
\usepackage{float}
\usepackage{framed}
\usepackage{array}
\usepackage{stmaryrd}
\usepackage{extarrows}
\usepackage{caption}
\captionsetup[figure]{labelfont={bf},name={Fig.},labelsep=period}

\theoremstyle{definition}
\newtheorem{df}{Definition}
\newtheorem{exa}{Example}
\newtheorem{ques}{Question}
\newtheorem{exr}{Exercise}
\newtheorem*{note}{Note}
\theoremstyle{plain}
\newtheorem{thm}{Theorem}
\newtheorem{prop}{Proposition}
\newtheorem{conj}{Conjecture}
\newtheorem{cor}{Corollary}
\newtheorem{lm}{Lemma}
\newtheorem*{fact}{Fact}
\newtheorem*{idea}{Idea}
\newtheorem*{clm}{Claim}
\newtheorem*{rmk}{Remark}
\usepackage[ruled]{algorithm2e}

\usepackage{ulem}
\makeatletter

\def\lf{\left\lfloor}   
\def\rf{\right\rfloor}
\def\lc{\left\lceil}   
\def\rc{\right\rceil}
\def\st{\text{ s.t. }}
\def\1{^{-1}}
\def\ind{\mathbf{1}}
\def\R{\mathbb{R}}
\def\Q{\mathbb{Q}}
\def\Z{\mathbb{Z}}
\def\C{\mathbb{C}}
\def\I{\mathbb{I}}
\def\N{\mathbb{N}}
\def\F{\mathbb{F}}
\def\A{\mathbb{A}}
\def\Li{\text{Li}}
\def\th{^\text{th}}
\def\sp{\text{Sp}}
\def\opn{\left\{}
\def\cls{\right\}}
\def\Aut{\text{Aut}}
\def\PG{\text{PG}}
\def\GL{\text{GL}}
\def\PGL{\text{PGL}}
\def\Cov{\text{Cov}}
\def\Pack{\text{Pack}}
\def\PgamL{\text{P}\Gamma\text{L}}
\def\gamL{\Gamma\text{L}}
\def\cl{\text{cl}}
\def\stbar{\ \middle\vert\ }
\def\partdone{\hphantom{1} \hfill \(\triangle\)}
\def\s0{_0}
\def\s1{_1}
\def\s2{_2}
\def\id{\mathrm{id}}
\def\topn{\text{ open}}
\def\Bd{\text{Bd }}
\renewcommand{\P}{\mathbb{P}}
\newcommand{\leg}[2]{\left( \frac{#1}{#2} \right)}
\renewcommand*\env@matrix[1][*\c@MaxMatrixCols c]{%
   \hskip -\arraycolsep
   \let\@ifnextchar\new@ifnextchar
   \array{#1}}
\makeatother

% These two lines suppress the warning generated 
% by amsmath for overwriting the choose command  
% because it's annoying. This probably has unint-
% ended ramifications somewhere else, but I'm too
% lazy to actually figure that out, so we'll cro-
% ss that bridge when we come to it lol.
\renewcommand{\choose}[2]{\left( {#1 \atop #2} \right)}
\WarningFilter{amsmath}{Foreign command} 

\renewcommand{\mod}[1]{\ (\mathrm{mod}\ #1)}
\renewcommand{\vec}[1]{\mathbf{#1}}

\let\oldprime\prime
\def\prime{^\oldprime}

\usepackage{float}
\restylefloat{figure}

\usepackage{cleveref}
\Crefname{thm}{Theorem}{Theorems}

% Comment commands for co-authors
\newcommand{\kmd}[1]{{\color{purple} #1}}

\newcolumntype{L}{>{$}l<{$}}
% Bib matter
\let\oldepsilon\epsilon
\def\epsilon{\varepsilon}

\let\oldphi\phi
\def\phi{\varphi}

%%% Local Variables:
%%% mode: plain-tex
%%% TeX-master: t
%%% End:

\graphicspath{{./img/}}

\begin{document}
\noindent \textbf{Math 171} \hfill \textbf{Professor Sebastian Bozlee} \\
\textbf{Scribed by: Kyle Dituro} \hfill \textbf{April 25, 2023}\hrule
\vspace{.2in}

\begin{framed}
  The final will be take home, posted on Gradescope Thursday, May 11th.

  There will be a 24 hour window. The exam will be open notes, open book, closed other people, closed internet.

  \(\frac{1}{3}\) should be pre-midterm content, the remainder being new.

  
\end{framed}
\begin{df}
  A topological space \(X\) is \textbf{compact} if each open cover of \(X\) admits a finite subcover.
\end{df}

\begin{thm}
  A closed subspace of a compact space is compact.
\end{thm}
\begin{proof}
  Let \(X\) be a compact topological space, and let \(Z \subseteq X\) be a closed subspace. Recall that \(Z\) is compact iff every collection of open sets \(\opn U_i \cls\) in \(X\) with union containing \(Z\) has a finite subcollection with the same property.

  Let \(\opn U_i \cls_{i \in I}\) be a collection of open subsets of \(X\) such that \(\bigcup_{i \in I} U_i \supseteq Z\). Notice now that since \(Z\) is closed, \(\opn U_i \cls_{i \in I} \cup \opn X - Z \cls\)  is an open cover of \(X\). Since \(X\) is compact, there exists \(i_1, \ldots, i_n\) such that \(\bigcup_{j = 1}^n U_{i_j} \cup (X - Z) = X\).

  Then \(\bigcup_{j = 1}^n U_{i_j} \supseteq Z\), and thus \(Z\) is compact.
\end{proof}

\begin{thm}
  Let \(X\) be a Hausdorff space, and let \(Z \subseteq X\) be a compact subspace. Then \(Z\) is closed inside of \(X\). 
\end{thm}
\begin{proof}
  Let \(X\) be a Hausdorff space, \(Z\) a compact subspace. We want to show that \(Z \supseteq \overline{Z}\). (Then \(Z = \overline{Z}\), so \(Z\) is closed).

  We'll show if \(y \in X - Z\), then \(y \not\in \overline{Z}\).

  Let \(y \in X - Z\). Then for each point \(x \in Z\), we can find disjoint open neighborhoods \(U_x\) of \(x\) and \(V_x\) of \(y\). Observe \(\opn U_x \cls_{x \in X}\) are a collection of open sets whose union contains \(Z\). Since \(Z\) is compact, there exists some finite subcover: there are \(x_1, \ldots, x_n\) such that \(U_{x_1} \cup \ldots \cup U_{x_n} \supseteq Z\). Then observe that \(V_{x_1} \cap \ldots \cap V_{x_n}\) is an open neighborhood of \(y\) disjoint from \(U_{x_1} \cup \ldots \cup U_{x_n}\).

  This implies \(\left( V_{x_1} \cap \ldots \cap V_{x_n}\right) \cap Z = \emptyset\). Thus \(y \not\in \overline{Z}\). Then \(Z\) is closed, as desired.
\end{proof}

\begin{exa}
  \((0,1]\) is not compact. A quick way to show this is to note \((0,1]\) is not a closed subspace of \(\R\).
\end{exa}

\begin{thm}
  Let \(f: X \to Y\) be a continuous function. If \(Z\) is a compact subspace of \(X\), then \(f(Z)\) is a compact subspace of \(Y\).
\end{thm}
\begin{proof}
  We can assume \(X = Z\).

  Then we want to show \(f(X)\) is compact. Let \(\opn U_i \cls_{i \in I}\) be a collection of open subsets of \(Y\) with \(\bigcup_{i \in I} U_i \supseteq f(X)\).

  Taking pre-images, \(\opn f\1(U_i) \cls_{i \in I}\) is a collection of open subsets of \(X\) with union.

  \begin{align*}
    \bigcup_{i \in I} f\1(U_i) = f\1\left(\bigcup_{i \in I} U_i\right) \supseteq f\1(f(X)) = X.
  \end{align*}

  Since \(X\) is compact, there are indices \(i_1, \ldots, i_n\) such that \(\bigcup_{j = 1}^n f\1(U_{i_j}) = X\). Then we claim \(\bigcup_{j = 1}^n U_{i_j} \supseteq f(X)\).

  Suppose for the sake of contradiction that there's some \(x \in X\) such that \(f(x) \not\in U_{i_j}\) for all \(j = 1, \ldots, n\). But this implies \(x \not\in f\1(U_{i_j})\) for all \(j = 1, \ldots, n\) \nope
\end{proof}

\begin{cor}
  If \(f:X \to Y\) is a continuous function from a compact space to a Hausdorff space, then \(f\) is a closed map.
\end{cor}

\begin{proof}
  Suppose \(Z \subseteq X\) is closed. Then \(Z\) is compact. Then \(f(Z)\) is compact. Then \(f(Z)\) is closed.
\end{proof}
\begin{cor}
  If \(f:X \to Y\) is bijective continuous function from a compact space to a Hausdorff space, then \(f\) is a homeomorphism.
\end{cor}

\begin{exa}
  Let \(X = S^1\). consider \(\sim\), \(\vec{X} \sim \pm \vec{x}\)

  \begin{center}
    \begin{tikzcd}
      S^1 \arrow[dr, "g"]& \\
      S^1/\opn \pm \cls \arrow[leftarrow, "p", u] & S^1 \arrow[dashleftarrow, l]
    \end{tikzcd}
  \end{center}

  So by corollary \(S^1/\opn \pm \cls \cong S^1\).
\end{exa}
\begin{framed}
  ``Compact + Hausdorff = Super closed''.

  If \(i: X \hookrightarrow X\) is the inclusion of a compact subspace to be a Hausdorff space, then \(Z\prime \subseteq Z\) is closed iff \(i(Z\prime)\) in \(X\) is closed.
\end{framed}
\begin{thm}
  If \(X\) and \(Y\) are compact subspaces, then \(X \times Y\) is also compact.
\end{thm}

\begin{lm}[Tube Lemma]
  If \(X\) is a topological space and \(Y\) is a compact space and \(Y\) is a compact space. \(x_0 \in X\) and \(Y \subseteq X \times Y\) is an open subset containing \(X_0 \times Y\), then there is an open neighborhood \(W\) of \(x_0\) such that \(W \times Y \subseteq N\).
\end{lm}
\end{document}
%%% Local Variables:
%%% mode: latex
%%% TeX-master: t
%%% End:

\documentclass[12pt, twosided]{article}

\usepackage[letterpaper,bindingoffset=0in,%
            left=1in,right=1in,top=1in,bottom=1in,%
            footskip=.25in]{geometry}

\usepackage{mathtools}
\usepackage{graphicx}

\usepackage{setspace}
\setstretch{1.1}

\usepackage{amsmath}
\usepackage{amsfonts}
\usepackage{amsthm}
\usepackage{amssymb}
\usepackage{csquotes}
\usepackage{relsize}

\usepackage{tikz}
\usetikzlibrary{cd}
\usetikzlibrary{fit,shapes.geometric}
\tikzset{%  
    mdot/.style={draw, circle, fill=black},
    mset/.style={draw, ellipse, very thick},
}

\usepackage{hhline}
\usepackage{systeme}
\usepackage{mathrsfs}
\usepackage{hyperref}
\usepackage{mathtools}  
\usepackage{silence}
\usepackage{blkarray}
\usepackage{float}
\usepackage{framed}
\usepackage{array}
\usepackage{stmaryrd}
\usepackage{extarrows}
\usepackage{caption}
\captionsetup[figure]{labelfont={bf},name={Fig.},labelsep=period}

\theoremstyle{definition}
\newtheorem{df}{Definition}
\newtheorem{exa}{Example}
\newtheorem{ques}{Question}
\newtheorem{exr}{Exercise}
\newtheorem*{note}{Note}
\theoremstyle{plain}
\newtheorem{thm}{Theorem}
\newtheorem{prop}{Proposition}
\newtheorem{conj}{Conjecture}
\newtheorem{cor}{Corollary}
\newtheorem{lm}{Lemma}
\newtheorem*{fact}{Fact}
\newtheorem*{idea}{Idea}
\newtheorem*{clm}{Claim}
\newtheorem*{rmk}{Remark}
\usepackage[ruled]{algorithm2e}

\usepackage{ulem}
\makeatletter

\def\lf{\left\lfloor}   
\def\rf{\right\rfloor}
\def\lc{\left\lceil}   
\def\rc{\right\rceil}
\def\st{\text{ s.t. }}
\def\1{^{-1}}
\def\ind{\mathbf{1}}
\def\R{\mathbb{R}}
\def\Q{\mathbb{Q}}
\def\Z{\mathbb{Z}}
\def\C{\mathbb{C}}
\def\I{\mathbb{I}}
\def\N{\mathbb{N}}
\def\F{\mathbb{F}}
\def\A{\mathbb{A}}
\def\Li{\text{Li}}
\def\th{^\text{th}}
\def\sp{\text{Sp}}
\def\opn{\left\{}
\def\cls{\right\}}
\def\Aut{\text{Aut}}
\def\PG{\text{PG}}
\def\GL{\text{GL}}
\def\PGL{\text{PGL}}
\def\Cov{\text{Cov}}
\def\Pack{\text{Pack}}
\def\PgamL{\text{P}\Gamma\text{L}}
\def\gamL{\Gamma\text{L}}
\def\cl{\text{cl}}
\def\stbar{\ \middle\vert\ }
\def\partdone{\hphantom{1} \hfill \(\triangle\)}
\def\s0{_0}
\def\s1{_1}
\def\s2{_2}
\def\id{\mathrm{id}}
\def\topn{\text{ open}}
\def\Bd{\text{Bd }}
\renewcommand{\P}{\mathbb{P}}
\newcommand{\leg}[2]{\left( \frac{#1}{#2} \right)}
\renewcommand*\env@matrix[1][*\c@MaxMatrixCols c]{%
   \hskip -\arraycolsep
   \let\@ifnextchar\new@ifnextchar
   \array{#1}}
\makeatother

% These two lines suppress the warning generated 
% by amsmath for overwriting the choose command  
% because it's annoying. This probably has unint-
% ended ramifications somewhere else, but I'm too
% lazy to actually figure that out, so we'll cro-
% ss that bridge when we come to it lol.
\renewcommand{\choose}[2]{\left( {#1 \atop #2} \right)}
\WarningFilter{amsmath}{Foreign command} 

\renewcommand{\mod}[1]{\ (\mathrm{mod}\ #1)}
\renewcommand{\vec}[1]{\mathbf{#1}}

\let\oldprime\prime
\def\prime{^\oldprime}

\usepackage{float}
\restylefloat{figure}

\usepackage{cleveref}
\Crefname{thm}{Theorem}{Theorems}

% Comment commands for co-authors
\newcommand{\kmd}[1]{{\color{purple} #1}}

\newcolumntype{L}{>{$}l<{$}}
% Bib matter
\let\oldepsilon\epsilon
\def\epsilon{\varepsilon}

\let\oldphi\phi
\def\phi{\varphi}

%%% Local Variables:
%%% mode: plain-tex
%%% TeX-master: t
%%% End:

\graphicspath{{./img/}}

\begin{document}
\noindent \textbf{Math 171} \hfill \textbf{Professor Sebastian Bozlee} \\
\textbf{Scribed by: Kyle Dituro} \hfill \textbf{\today}\hrule
\vspace{.2in}

\begin{thm}[Universal Property of the Quotient]
  Let \(p: X\ to Y\) be a quotient map. Then for any continuous function \(g: X \to Z\) satisfying \(p(x_1) = p(x_2) \Rightarrow g(x_1) = g(x_2)\), there exists a unique function \(f: Y \to Z\) such that
  \begin{center}
    \begin{tikzcd}
      X \arrow[dr, "q"] & \\
      Y \arrow[leftarrow, u, "p"]& Z \arrow[leftarrow, l, "f"]
    \end{tikzcd}
  \end{center}
\end{thm}

\begin{proof}
  We want to show that this \(f\) is well-defined, unique, makes the diagram commute, and is continuous.

  For well-definedness, we need that each element \(y \in Y\) can be written as \(p(x)\) for some \(x \in x\) and if \(p(x_1) = p(x_2)\) for some \(x_1, x_2 \in X\) we have that \(g(x_1) = g(x_2)\). The first part is true since \(p\) is surjective, and the second part is true by hypothesis.

  in order for the diagram to commute, we need
  \begin{align*}
    f(p(x)) = g(x) \quad \text{for all } x\in X,
  \end{align*}
  Which is true by construction, so \(f(p(x)) = g(x)\) fully determines \(f\), so we have uniqueness.

  To check that \(f\) is continuous, suppose \(U \subseteq Z\) is an open subset. Then since \(g\) is continuous, \(g\1(U) \subseteq X\) is open. Then \(g\1(U) = p\1(f\1(U))\) is open in \(X\).

  Then by definition of the quotient map, \(f\1(U) \subseteq Y\) is open, as desired.
\end{proof}

\begin{exa}[Quotients need not behave well with respect to subspaces]
  Write \(\R^\delta\) for \(\R\) with the discrete topology, and \(\R^i\) for \(\R\) with the indiscrete topology.

  Say
  \begin{align*}
    X = \R^\delta \sqcup \R^i = \opn (1, x) \stbar x \in \R \cls \cup \opn (2, x) \stbar x \in \R \cls.
  \end{align*}

  Which will gain the union topology: the open sets are those of the form \((1 \times U) \cup (2 \times V)\) where \(U \subseteq \R^\delta\) ad \(V \subseteq \R^i\) are open.

  Consider \(p
  \begin{matrix}
    \pi: X \to \R^i \\ (1, x) \mapsto x \\ (2, x) \mapsto x
  \end{matrix}.
  \), which we claim to be a quotient map.

  % \begin{center}
  %   \begin{tikzpicture}
  %     \draw (-1,0) -- (1,0) \node[anchor=right]{\(1\ times \R\)};
  %   \end{tikzpicture}
  % \end{center}

  Let \(B \subseteq \R^i\), then \(\pi\1(B) = (1 \times B)\cup (2 \times B)\) is open if an only if \(B = \emptyset, \R\)

  But! If we restrict \(\pi\) to \(\R^\delta\), then we get \(
  \begin{matrix}

    R^\delta \to \R^i \\ (1, x) \mapsto x
  \end{matrix}
  \) which is not a quotient, since too many pre-images are open.
\end{exa}

\begin{lm}
  Let \(p:X \to Y\) be a quotient map, \(B \subseteq Y\) is a subset, \(A = p\1(b)\) be its preimage. If \(B\) is open (or closed), then the restriction \(q = p\vert_A: A \to B\) is also a quotient map. (e.g. \(\widetilde{D}(z) \to D(z) \) from this weeks recitation is an example).
\end{lm}

\begin{proof}
  Note that if \(V \subseteq B\) then \(q\1(V) = p\1(V)\) . Assume that \(A\) is open, and let \(V \subseteq B\) be a subset. We know that \(q\) is continuous and \(q\) is surjective by definition. All that remains to be shown is that \(q\1(V) \subseteq A\) is open, so \(V \subseteq B\) is open.

  Well \(q\1(V) = p\1(V)\) since \(q\1(V) \subseteq A\) is open and \(A \subseteq X\) is open, \(q\1(V) = p\1(V)\) is open in \(X\).

  Since \(p\) is a quotient map, \(V \subseteq Y\) is open. Since \(V \subseteq B\) and \(B \subseteq Y\) is open, \(V \subseteq B\) is open.

  For the case where \(B\) is closed, replace ``open'' with ``closed'' throughout.
\end{proof}

\begin{exa}[Hausdorff not preserved by quotients]
  Let \(X = \R \sqcup \R\), and take the equivalence relation
  \begin{align*}
    (1, x) \sim (2, x) \quad \text{when } x \neq 0.
  \end{align*}

  And consider \(\pi: X \to X/\sim\). This construction is called the ``bug-eyed line'' or the ``line with two origins''

  Let \(x = [(1, 0)], y = [(2,0)]\). Suppose that \(U\)  is an open neighborhood of \(x\) and \(V\) is an open neighborhood of \(Y\).

  Well since \(U\) is open, \(\pi\1(U)\) is open. Then for any arbitrary set, its pre-image will be the two copies of \(U\) which live in each copy of \(\R\) everywhere but at \(0\).

  Notice that \(U\) and \(V\) necessarily intersect, so \(X/ \sim\) is not Hausdorff.
\end{exa}

\begin{ques}
  What if \(\sim\) was \((1,x) \sim \left(2, \frac{1}{x}\right)\) instead? What would \(X/ \sim\) be?
\end{ques}

\begin{lm}
  Let \(p: X \to Y\) be a quotient map. \(q: Y \to Z\) be a function. Then \(q\) is a quotient map if and only if \(q \circ p\) is a quotient map.
\end{lm}

\begin{proof}
  Suppose that \(q\) is a quotient map. then \(q \circ p\) is continuous and surjective. Then 

  \begin{align*}
    B \subseteq Z \topn &\Leftrightarrow q\1(B) \subseteq Y \topn \\
                        &\Leftrightarrow p\1(q\1(B)) \subseteq X \topn \\
                        &\Leftrightarrow (q \circ p)\1 (B) \subseteq X \topn.
  \end{align*}\partdone

  Conversely, suppose \(q \circ p: X \to Z\) is a quotient map. Then since \(q \circ p\) is surjective, \(q\) is surjective. Suppose \(B \subseteq Z\). Then

  \begin{align*}
    B \subseteq Z \topn &\Leftrightarrow (q \circ p)\1(B) \subseteq X \topn \\
                        &\Leftrightarrow p\1(q\1(B)) \subseteq X \topn \\
                        &\Leftrightarrow q\1(B) \subseteq Y \topn.                
  \end{align*}
  Therefore \(q\) is a quotient map.
\end{proof}

\begin{exa}
  Let \(X = \R^3 - \opn 0 \cls\), \(\sim\) be the relation
  \begin{align*}
    \vec{x} \sim \vec{y} \Longleftrightarrow \exists c \in \R - \opn 0 \cls \st c\vec{x} = \vec{y}.
  \end{align*}

  Then \(\pi: X \to X/\sim\ = \R\mathbb{P}^2\).

  Let \[
  \begin{matrix}
    p: X \to \mathbb{S}^2 = \opn (x, y, z) \in \R^3 \stbar x^2 + y^2 + z^2 =1 \cls \\ \vec{x} \mapsto \frac{1}{|\vec{x}|}\vec{x}
  \end{matrix}.
  \]

  Believe that this is a quotient map.

  Define \(
  \begin{matrix}
q : \mathbb{S}^2 \to \R\mathbb{P}^2 \\ \vec{x} \mapsto [x:y:z]
  \end{matrix}
  \).

  The composite \(q \circ p = \pi\).
\end{exa}
\end{document}
%%% Local Variables:
%%% mode: latex
%%% TeX-master: t
%%% End:

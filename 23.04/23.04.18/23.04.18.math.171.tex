\documentclass[12pt, twosided]{article}

\usepackage[letterpaper,bindingoffset=0in,%
            left=1in,right=1in,top=1in,bottom=1in,%
            footskip=.25in]{geometry}

\usepackage{mathtools}
\usepackage{graphicx}

\usepackage{setspace}
\setstretch{1.1}

\usepackage{amsmath}
\usepackage{amsfonts}
\usepackage{amsthm}
\usepackage{amssymb}
\usepackage{csquotes}
\usepackage{relsize}

\usepackage{tikz}
\usetikzlibrary{cd}
\usetikzlibrary{fit,shapes.geometric}
\tikzset{%  
    mdot/.style={draw, circle, fill=black},
    mset/.style={draw, ellipse, very thick},
}

\usepackage{hhline}
\usepackage{systeme}
\usepackage{mathrsfs}
\usepackage{hyperref}
\usepackage{mathtools}  
\usepackage{silence}
\usepackage{blkarray}
\usepackage{float}
\usepackage{framed}
\usepackage{array}
\usepackage{stmaryrd}
\usepackage{extarrows}
\usepackage{caption}
\captionsetup[figure]{labelfont={bf},name={Fig.},labelsep=period}

\theoremstyle{definition}
\newtheorem{df}{Definition}
\newtheorem{exa}{Example}
\newtheorem{ques}{Question}
\newtheorem{exr}{Exercise}
\newtheorem*{note}{Note}
\theoremstyle{plain}
\newtheorem{thm}{Theorem}
\newtheorem{prop}{Proposition}
\newtheorem{conj}{Conjecture}
\newtheorem{cor}{Corollary}
\newtheorem{lm}{Lemma}
\newtheorem*{fact}{Fact}
\newtheorem*{idea}{Idea}
\newtheorem*{clm}{Claim}
\newtheorem*{rmk}{Remark}
\usepackage[ruled]{algorithm2e}

\usepackage{ulem}
\makeatletter

\def\lf{\left\lfloor}   
\def\rf{\right\rfloor}
\def\lc{\left\lceil}   
\def\rc{\right\rceil}
\def\st{\text{ s.t. }}
\def\1{^{-1}}
\def\ind{\mathbf{1}}
\def\R{\mathbb{R}}
\def\Q{\mathbb{Q}}
\def\Z{\mathbb{Z}}
\def\C{\mathbb{C}}
\def\I{\mathbb{I}}
\def\N{\mathbb{N}}
\def\F{\mathbb{F}}
\def\A{\mathbb{A}}
\def\Li{\text{Li}}
\def\th{^\text{th}}
\def\sp{\text{Sp}}
\def\opn{\left\{}
\def\cls{\right\}}
\def\Aut{\text{Aut}}
\def\PG{\text{PG}}
\def\GL{\text{GL}}
\def\PGL{\text{PGL}}
\def\Cov{\text{Cov}}
\def\Pack{\text{Pack}}
\def\PgamL{\text{P}\Gamma\text{L}}
\def\gamL{\Gamma\text{L}}
\def\cl{\text{cl}}
\def\stbar{\ \middle\vert\ }
\def\partdone{\hphantom{1} \hfill \(\triangle\)}
\def\s0{_0}
\def\s1{_1}
\def\s2{_2}
\def\id{\mathrm{id}}
\def\topn{\text{ open}}
\def\Bd{\text{Bd }}
\renewcommand{\P}{\mathbb{P}}
\newcommand{\leg}[2]{\left( \frac{#1}{#2} \right)}
\renewcommand*\env@matrix[1][*\c@MaxMatrixCols c]{%
   \hskip -\arraycolsep
   \let\@ifnextchar\new@ifnextchar
   \array{#1}}
\makeatother

% These two lines suppress the warning generated 
% by amsmath for overwriting the choose command  
% because it's annoying. This probably has unint-
% ended ramifications somewhere else, but I'm too
% lazy to actually figure that out, so we'll cro-
% ss that bridge when we come to it lol.
\renewcommand{\choose}[2]{\left( {#1 \atop #2} \right)}
\WarningFilter{amsmath}{Foreign command} 

\renewcommand{\mod}[1]{\ (\mathrm{mod}\ #1)}
\renewcommand{\vec}[1]{\mathbf{#1}}

\let\oldprime\prime
\def\prime{^\oldprime}

\usepackage{float}
\restylefloat{figure}

\usepackage{cleveref}
\Crefname{thm}{Theorem}{Theorems}

% Comment commands for co-authors
\newcommand{\kmd}[1]{{\color{purple} #1}}

\newcolumntype{L}{>{$}l<{$}}
% Bib matter
\let\oldepsilon\epsilon
\def\epsilon{\varepsilon}

\let\oldphi\phi
\def\phi{\varphi}

%%% Local Variables:
%%% mode: plain-tex
%%% TeX-master: t
%%% End:

\graphicspath{{./img/}}

\begin{document}
\noindent \textbf{Math 171} \hfill \textbf{Professor Sebastian Bozlee} \\
\textbf{Scribed by: Kyle Dituro} \hfill \textbf{\today}\hrule
\vspace{.2in}

\begin{df}
  Given a topological space \(X\), define an equivalence relation on it via the rule \(x \sim y\) iff there is a connected subspace of \(X\) containing both of those points. The equivalence classes of \(\sim\) are the \textbf{connected components} (or just `components') of \(X\).
\end{df}
\begin{proof}
  A quick check shows that \(\sim\) does indeed form an equivalence relation.

  \(x \sim x\) trivially, as does \(x \sim y \Rightarrow y \sim x\).

  The more interesting one is transitivity.

  If \(x \sim y\), \(y \sim z\), then there exit connected subspaces \(A, B \) of \(X\) such that \(x, y \in A\), \(y, z \in B\). Then the union of \(A\) and \(B\) gives us a connected component since \(y\) is a common point.
\end{proof}

\begin{exa}
  If \(X\) is connected, then \(X\) is its only connected component.

  If \(X = \Q\), then the connected components are just the points.
\end{exa}

\begin{lm}
  The connected components of \(X\) are connected disjoint subspaces of \(X\) whose union is \(X\) such that each nonempty connected subspace of \(X\) intersects only one of them.
\end{lm}

\begin{proof}
  Since the connected components of \(X\) are equivalence classes, it is clear that they are disjoint with union \(X\). Let's show that a connected component \(C\) of \(X\) truly is connected.

  Let \(x_0\) be a point of \(C\).  If \(x \in C\) is any point, then by definition that \(x_0 \sim x\), so there exists a connected subspace \(A_X\) of \(X\) such that \(x_0, x \in A_X\). Then \(C\) is the union of the \(A_X\)'s and the \(A_X\)'s have the point \(x_0\) in common. Therefore \(C\) is connected.

  Finally, let's show that a nonempty connected subspace \(A \subseteq X\) intersects exactly one connected component of \(X\).

  Suppose that \(C_1, C_2\) are two connected components intersecting \(A\), in respective points \(x_1\) and \(x_2\). Then by definition, \(x_1 \sim x_2\), since both belong to \(A\). But then if \(x_1 \sim x_2\), then \(C_1 = C_2\). 
\end{proof}

Importantly, connected components are always closed, but not necessarily open. However, if there are finitely many connected components, then they are clopen.

We won't prove this fact explicitly, but intuitively, if you take a separation of the space, then the connected component must lie in one portion of the separation. Then taking the intersection of all such separations containing our component, we get that our component is the intersection of (potentially infinitely many) closed sets, and is thus closed.

\begin{df}
  Let \(X\) be a topological space. Recall that a collection of open subsets \(\opn U_i \cls_{i \in I}\) of \(X\) is said to be an \textbf{open cover} of \(X\) if \(X = \bigcup_{i \in I} U_i\). A subcollection of \(\opn U_i \cls_{i \in I}\) is said to be a subcover if it is still a cover. It is said to be a finite subcover if there are only finitely many sets in the subcover.

  More formally, a finite subcover is determined by indices \(\alpha_1,\ldots, \alpha_n\) such that \[\bigcup_{i = 1}^n U_{\alpha_i} = X\].
\end{df}

\begin{df}
\(X\) is said to be \textbf{compact} if every open cover of \(X\) has a finite subcover.
\end{df}

\begin{exa}
  Take \(X = [0,1]\). Consider the open cover:
  \begin{align*}
    U_0 &= \left(\frac{1}{3}, 1\right] \\
    U_n &= \left[0,1-\frac{1}{e^n}\right) \text{ for each } n= 1,2,\ldots
  \end{align*}

  Then \(U_{100} \sup U_0\) is a finite subcover.
\end{exa}

\begin{fact}
  \([0, 1]\) is compact. This will be proven later.
\end{fact}

\begin{exa}
  Take \(X = [0,1)\), and consider the open cover

  \begin{align*}
    U_n = \left[0, 1- \frac{1}{e^n}\right) \quad n = 1, 2,\ldots
  \end{align*}

  This is an open cover, but it has no finite subcover. If it did, then there would be a max \(N\) included, and we'd have \(\bigcup U_\alpha = \left[0, 1 - \frac{1}{e^N}\right) \subsetneq [0,1)\), obviously a contradiction. So \([0, 1)\) is not compact.
\end{exa}

\begin{exa}
  \(\R\). Then \(U_n = (n, n+2)\) for \(n \in \Z\) form an open cover, which obviously has no finite subcover. So \(\R\) is not compact.
\end{exa}

\begin{fact}
  \(\R^n\) with the Zariski topology is compact. Thank God.
\end{fact}

\begin{df}
  If \(Y\) is a subspace of \(X\), we say a collection of subsets of \(X\) is a cover of \(Y\) if its union contains \(Y\).
\end{df}

\begin{lm}
  Let \(Y\) be a subspace of \(X\). Then \(Y\) is compact if and only if every open cover of \(Y\) by open subsets of \(X\) has a finite subcover also containing \(Y\).
\end{lm}
\begin{proof}
  \begin{enumerate}
  \item [(\(\Rightarrow\))] Suppose \(Y\) is compact, and let \(\opn U_i \cls_{i \in I}\) be a collection of open subsets of \(X\) such that \(\bigcup_{i \in I} U_i \supseteq Y\).

    Then \(\opn V_i \cls\), where \(V_i = U_i \cap Y\) is an open cover of \(Y\) in the subspace topology:

    \begin{align*}
      \bigcup_{i \in I} V_i = \bigcup_{i \in I} (U_i \cap Y) = \left(\bigcup_{i \in I} U_i \right) \cap Y = Y.
    \end{align*}

    Then since \(Y\) is compact, there is a finite lest of indices \(\alpha_1, \ldots, \alpha_n\) such that \(\bigcup_{j = 1}^n V_{\alpha_j} = Y\). Then \[\bigcup_{j = 1}^n V_{\alpha_j} = \left(\bigcup_{j = 1}^n U_{\alpha_j}\right) \cap Y = Y,\] so \(\bigcup_{j = 1}^nU_{\alpha_j} \supseteq Y\).
  \item [(\(\Leftarrow\))] Suppose that every open cover of \(Y\) by open subsets of \(X\) has a finite subcover. We want to show that \(Y\) is compact. Let \(\opn V_i \cls_{i \in I}\) be an open cover of \(Y\).

    By definition of the subspace topology, for each \(i \in I\), there is an open subset \(U_i\) of \(X\) such that \(U_i \cap Y = V_i\). Then \(Y = \bigcup_{i \in I} V_i = \left( \bigcup_{i \in I} U_i\right) \cap Y\), so \(\bigcup_{i \in I} U_i \supseteq Y\). Then \(\opn U_i \cls\) is a cover of \(Y\) by open subsets of \(X\), so there exists \(\alpha_1, \ldots \alpha_n\) such that \(\bigcup_{j = 1}^n U_{\alpha_j} \supseteq Y\). Then \(\bigcup_{j = 1}^n V_{\alpha_j} = \left( \bigcup_{j = 1}^nU_{\alpha_j}\right) \cap Y = Y\).
  \end{enumerate}
\end{proof}
\end{document}
%%% Local Variables:
%%% mode: latex
%%% TeX-master: t
%%% End:

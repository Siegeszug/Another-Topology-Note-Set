\documentclass[12pt, twosided]{article}

\usepackage[letterpaper,bindingoffset=0in,%
            left=1in,right=1in,top=1in,bottom=1in,%
            footskip=.25in]{geometry}

\usepackage{mathtools}
\usepackage{graphicx}

\usepackage{setspace}
\setstretch{1.1}

\usepackage{amsmath}
\usepackage{amsfonts}
\usepackage{amsthm}
\usepackage{amssymb}
\usepackage{csquotes}
\usepackage{relsize}

\usepackage{tikz}
\usetikzlibrary{cd}
\usetikzlibrary{fit,shapes.geometric}
\tikzset{%  
    mdot/.style={draw, circle, fill=black},
    mset/.style={draw, ellipse, very thick},
}

\usepackage{hhline}
\usepackage{systeme}
\usepackage{mathrsfs}
\usepackage{hyperref}
\usepackage{mathtools}  
\usepackage{silence}
\usepackage{blkarray}
\usepackage{float}
\usepackage{framed}
\usepackage{array}
\usepackage{stmaryrd}
\usepackage{extarrows}
\usepackage{caption}
\captionsetup[figure]{labelfont={bf},name={Fig.},labelsep=period}

\theoremstyle{definition}
\newtheorem{df}{Definition}
\newtheorem{exa}{Example}
\newtheorem{ques}{Question}
\newtheorem{exr}{Exercise}
\newtheorem*{note}{Note}
\theoremstyle{plain}
\newtheorem{thm}{Theorem}
\newtheorem{prop}{Proposition}
\newtheorem{conj}{Conjecture}
\newtheorem{cor}{Corollary}
\newtheorem{lm}{Lemma}
\newtheorem*{fact}{Fact}
\newtheorem*{idea}{Idea}
\newtheorem*{clm}{Claim}
\newtheorem*{rmk}{Remark}
\usepackage[ruled]{algorithm2e}

\usepackage{ulem}
\makeatletter

\def\lf{\left\lfloor}   
\def\rf{\right\rfloor}
\def\lc{\left\lceil}   
\def\rc{\right\rceil}
\def\st{\text{ s.t. }}
\def\1{^{-1}}
\def\ind{\mathbf{1}}
\def\R{\mathbb{R}}
\def\Q{\mathbb{Q}}
\def\Z{\mathbb{Z}}
\def\C{\mathbb{C}}
\def\I{\mathbb{I}}
\def\N{\mathbb{N}}
\def\F{\mathbb{F}}
\def\A{\mathbb{A}}
\def\Li{\text{Li}}
\def\th{^\text{th}}
\def\sp{\text{Sp}}
\def\opn{\left\{}
\def\cls{\right\}}
\def\Aut{\text{Aut}}
\def\PG{\text{PG}}
\def\GL{\text{GL}}
\def\PGL{\text{PGL}}
\def\Cov{\text{Cov}}
\def\Pack{\text{Pack}}
\def\PgamL{\text{P}\Gamma\text{L}}
\def\gamL{\Gamma\text{L}}
\def\cl{\text{cl}}
\def\stbar{\ \middle\vert\ }
\def\partdone{\hphantom{1} \hfill \(\triangle\)}
\def\s0{_0}
\def\s1{_1}
\def\s2{_2}
\def\id{\mathrm{id}}
\def\topn{\text{ open}}
\def\Bd{\text{Bd }}
\renewcommand{\P}{\mathbb{P}}
\newcommand{\leg}[2]{\left( \frac{#1}{#2} \right)}
\renewcommand*\env@matrix[1][*\c@MaxMatrixCols c]{%
   \hskip -\arraycolsep
   \let\@ifnextchar\new@ifnextchar
   \array{#1}}
\makeatother

% These two lines suppress the warning generated 
% by amsmath for overwriting the choose command  
% because it's annoying. This probably has unint-
% ended ramifications somewhere else, but I'm too
% lazy to actually figure that out, so we'll cro-
% ss that bridge when we come to it lol.
\renewcommand{\choose}[2]{\left( {#1 \atop #2} \right)}
\WarningFilter{amsmath}{Foreign command} 

\renewcommand{\mod}[1]{\ (\mathrm{mod}\ #1)}
\renewcommand{\vec}[1]{\mathbf{#1}}

\let\oldprime\prime
\def\prime{^\oldprime}

\usepackage{float}
\restylefloat{figure}

\usepackage{cleveref}
\Crefname{thm}{Theorem}{Theorems}

% Comment commands for co-authors
\newcommand{\kmd}[1]{{\color{purple} #1}}

\newcolumntype{L}{>{$}l<{$}}
% Bib matter
\let\oldepsilon\epsilon
\def\epsilon{\varepsilon}

\let\oldphi\phi
\def\phi{\varphi}

%%% Local Variables:
%%% mode: plain-tex
%%% TeX-master: t
%%% End:

\graphicspath{{./img/}}

\usepackage{pgfplots}
\pgfplotsset{compat=newest}
\usetikzlibrary{decorations.markings}

\def\nframes{4}
\def\frame{0}

\begin{document}
\noindent \textbf{Math 171} \hfill \textbf{Professor Sebastian Bozlee} \\
\textbf{Scribed by: Kyle Dituro} \hfill \textbf{April 4, 2023}\hrule
\vspace{.2in}

We begin thinking about the quotient topology, which will give us the power to ``glue'' topological spaces.

The most common example is the famous transformation of the unit square into a torus (doughnut) by associating opposite sides of squares as equal. Namely, we identify
\begin{align*}
  (0, y) &\sim (1, y) \\
  (x, 0) &\sim (x, 1)
\end{align*}
in the space \(X = [0, 1] \times [0, 1]\).

\begin{center}
  \foreach \frame in {0,1,...,\nframes}
  {

    \pgfmathsetmacro{\time}{\frame / \nframes}
    \pgfmathsetmacro{\c}{20 + (3 - 20) / (1 + exp(-10 * (\time - 0.6)))}
    \pgfmathsetmacro{\a}{20 + (1 - 20) / (1 + exp(-8 * (\time - 0.3)))}
    \pgfmathsetmacro{\xrange}{3 + (180 - 3) / (1 + exp(-14 * (\time - 0.6)))}
    \pgfmathsetmacro{\yrange}{3 + (180 - 3) / (1 + exp(-10 * (\time - 0.3)))}
    \pgfmathsetmacro{\theta}{90 + (45 - 90) * \time}
    \pgfmathsetmacro{\phi}{0 + (25 - 0) * \time}

    \pgfplotsset{
      border one/.style={
        thick,
        red,
        samples y  = 0,
        variable   = \t,
        domain     = -\xrange:\xrange,
        postaction = {decorate},
        decoration = {markings,
          mark = at position 0.48 with {\arrow{stealth}},
          mark = at position 0.52 with {\arrow{stealth}}}
      },
      border two/.style={
        thick,
        green,
        samples y  = 0,
        variable   = \t,
        domain     = -\yrange:\yrange,
        postaction = {decorate},
        decoration = {markings, mark = at position 0.5 with {\arrow{stealth}}}
      }
    }



    \begin{tikzpicture}
      \useasboundingbox (0, 0) rectangle (6, 6);
      \begin{axis} [
        hide axis,
        view               = {\theta}{\phi},
        domain             = -\xrange:\xrange,
        y domain           = -\yrange:\yrange,
        samples            = 20,
        samples y          = 20,
        unit vector ratio  = 1 1 1,
        declare function   = {
          u(\x,\y) = (\c + \a * cos(\y)) * cos(\x);
          v(\x,\y) = (\c + \a * cos(\y)) * sin(\x);
          w(\x,\y) = \a * sin(\y);
        }
        ]

        \addplot3 [
        surf,
        color         = blue,
        opacity       = 0.01,
        faceted color = white,
        z buffer      = sort,
        fill opacity  = 0.5] ({u(\x, \y)}, {v(\x, \y)}, {w(\x, \y)});

        \addplot3 [border one] ({u(\t, \yrange)}, {v(\t, \yrange)}, {w(\t, \yrange)});
        \addplot3 [border one] ({u(\t, -\yrange)}, {v(\t, -\yrange)}, {w(\t, -\yrange)});
        \addplot3 [border two] ({u(\xrange, \t)}, {v(\xrange, \t)}, {w(\xrange, t)});
        \addplot3 [border two] ({u(-\xrange, \t)}, {v(-\xrange, \t)}, {w(-\xrange, t)});
      \end{axis}
    \end{tikzpicture}}
\end{center}
So really what we are doing is identifying equivalence classes, which immediately brings to mind quotients and their universal property:

\begin{thm}
  \(
  \begin{matrix}
    p: X \to X /\sim \\ x \mapsto [x]
  \end{matrix}
  \) {\color{red} a continuous function} has the following property:
  \begin{enumerate}
  \item if \(f: X/\sim \to Z\) is any {\color{red} continuous} function, \(f \circ p: X \to Z\) respects the equivalence relation.
  \item If \(g: X \to Z\) is any {\color{red} continuous} function that respects \(\sim\), then there exists a unique {\color{red} continuous} function \(f: X/\sim \to Z\) such that

    \begin{center}
      \begin{tikzcd}
        X \arrow[d, "p"] \arrow[dr, "g"] & \\
        X /\sim \arrow[dotted, r, "\exists ! f"] & Z 
      \end{tikzcd}
    \end{center}
  \end{enumerate}
\end{thm}

What should go in this topology?

Recall \(S = \opn 1, 2 \cls\) with the topology \(\tau_S = \opn \emptyset, \opn 1 \cls, S \cls\).

Then a continuous function \(f : X \to S\) is ``the same'' as an open subset of \(X\). In other words, given \(f\), \(f\1(\opn 1\cls)\) is an open set.

Assume that \(p: X \to X/\sim\) has the universal property. Then from part 2, for any function \(g: X \to S\) respecting \(\sim\), we should get a unique continuous \(f: X/\sim \to S\) making the diagram commute. Since \(g\) respects \(\sim\), if \(x \in g\1(\opn 1 \cls)\), we better have for any \(x_2 \sim x\sb1\) that \(x\sb2 \in g\1(\opn 1 \cls)\) .
i.e. the possible open subsets we get from \(g\1(\opn 1 \cls)\) are the open subsets with the property that \(x\sb1 \in U,\ x_1 \sim x_2 \Rightarrow x_2 \in U\).

So using the universal property, we should have an \(f: X/\sim \to S\) that makes \begin{tikzcd}[cramped, sep=small] X \arrow[d] \arrow[dr] & \\ X/\sim & S \arrow[leftarrow, l]\end{tikzcd} commute.

This function is defined by \(f([x]) = g(x)\).

\begin{clm}
  The open subsets we get from \(g\) are the preimages of the open subsets we get from \(f\). It follows that a subset \(U \subseteq X / \sim\) should be open if and only if \(p\1(U)\) is open
\end{clm}

\begin{df}
  A continuous function \(p: X \to Y\) of topological spaces is a \textbf{quotient map} if \(U \subseteq Y\) is open if and only iff \(p\1(U) \subseteq X\) is open.

  This can be stated equivalently using complements as:

  \begin{align*}
    Z \subseteq Y \text{ is closed} \Longleftrightarrow p\1(Z) \subseteq X \text{ is closed}.
  \end{align*}
\end{df}


\begin{fact}
  If \(p:X \to Y\) is surjective and \(B \subseteq Y\) then
  \begin{align*}
    p(p\1(B)) = B
  \end{align*}
  Notice that this is not necessarily true, and in fact is often not, if \(p\) is not surjective.
\end{fact}

\begin{df}
  A subset \(C\) of \(X\) is said to be \textbf{saturated} with respect to a function \(p: X \to Y\) if there exists a subset \(B\) of \(Y\) such that \(C = p\1(B)\).
\end{df}

\begin{lm}
  \(C \subseteq X\) is saturated with respect to \(p: X \to Y\) if and only if \(C\) contains every fiber \(p\1(\opn y \cls )\) that is intersects.
\end{lm}

\begin{proof}
  \begin{enumerate}
  \item [(\(\Rightarrow\))]Suppose that \(C\) is saturated, and let \(B \subseteq Y\) be the subset such that \(C =  p\1(B)\).

    Then if \(y \in y\), and \(C \cap p\1(\opn y\cls)\), then \(\exists c \in C\) such that \(p(c) = y\).

    Then \(y \in B\) so \(p\1(\opn y \cls) \subseteq p\1(B) = C\) by assumption \partdone
  \item [(\(\Leftarrow\))] Now suppose that \(C\) contains every fiber that it intersects. Consider \(B = p(C)\). Then since \(B = \bigcup_{b \in B} \opn b \cls\), \(p\1 B = \bigcup_{b \in B} p\1(\opn b \cls)\).
  \end{enumerate}

  Since \(p\1(\opn b \cls) \cap C \neq \emptyset\)  for all \(b \in B\), \(p\1(\opn b \cls) \subseteq C\).

  Then this implies that \(p\1(B) \subseteq C\).

  On the other hand, \(C \subseteq p\1(p(C))\), so \(C = p\1(B)\).
\end{proof}

Now with all of this done, we can finally re-write what a quotient map is.

\begin{align*}
p: X \to Y \text{is a quorient map} \Longleftrightarrow
  \begin{pmatrix}
    p \text{ is continuous, surjective, and} \\ \text{the image of each saturated} \\ \text{open set of } X \topn
  \end{pmatrix} \\
  \hspace{1em}\\
  \left(\forall U \subseteq Y, U\topn \right) \Leftrightarrow p(v) \text{ is open in } y \Leftrightarrow V \text{ is a saturated subset of } X \topn
\end{align*}

\begin{exa}
  \(\pi_1: X \times Y \to X\) is a quotient map.
\end{exa}

\begin{df}
  If \(X\) is a topological space and \(A\) is a set, and \(p: X \to A\) is a surjective function, there is a unique topology on \(A\) such that \(p\) becomes a quotient map. This is the \textbf{quotient topology} on \(X\).
  \begin{align*}
    U \subseteq A \topn \Leftrightarrow p\1(U) \subseteq X \topn
  \end{align*}
\end{df}
Let's now prove that this is a topology. This should be easy.
\begin{proof}
  \begin{enumerate}
  \item \(\emptyset \subseteq A\) is open since \(p\1(\emptyset) = \emptyset\) is open inp \(X\)
  \item \(A \subseteq A\) is open since \(p\1(A) = X\) is open in \(X\).
  \item If \(\opn U_i \cls\) are open, then \(p\left(\bigcup U_i \right) = \underbrace{\bigcup_{i \in I} \underbrace{p\1(U_i)}_{\text{open in } X}}_{\text{open in } X}\), so \(\bigcup_{i \in I} U_i\) is open.
  \end{enumerate}
\end{proof}
\end{document}
%%% Local Variables:
%%% mode: latex
%%% TeX-master: t
%%% End:

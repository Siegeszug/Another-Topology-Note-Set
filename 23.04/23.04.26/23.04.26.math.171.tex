\documentclass[12pt, twosided]{article}

\usepackage[letterpaper,bindingoffset=0in,%
            left=1in,right=1in,top=1in,bottom=1in,%
            footskip=.25in]{geometry}

\usepackage{mathtools}
\usepackage{graphicx}

\usepackage{setspace}
\setstretch{1.1}

\usepackage{amsmath}
\usepackage{amsfonts}
\usepackage{amsthm}
\usepackage{amssymb}
\usepackage{csquotes}
\usepackage{relsize}

\usepackage{tikz}
\usetikzlibrary{cd}
\usetikzlibrary{fit,shapes.geometric}
\tikzset{%  
    mdot/.style={draw, circle, fill=black},
    mset/.style={draw, ellipse, very thick},
}

\usepackage{hhline}
\usepackage{systeme}
\usepackage{mathrsfs}
\usepackage{hyperref}
\usepackage{mathtools}  
\usepackage{silence}
\usepackage{blkarray}
\usepackage{float}
\usepackage{framed}
\usepackage{array}
\usepackage{stmaryrd}
\usepackage{extarrows}
\usepackage{caption}
\captionsetup[figure]{labelfont={bf},name={Fig.},labelsep=period}

\theoremstyle{definition}
\newtheorem{df}{Definition}
\newtheorem{exa}{Example}
\newtheorem{ques}{Question}
\newtheorem{exr}{Exercise}
\newtheorem*{note}{Note}
\theoremstyle{plain}
\newtheorem{thm}{Theorem}
\newtheorem{prop}{Proposition}
\newtheorem{conj}{Conjecture}
\newtheorem{cor}{Corollary}
\newtheorem{lm}{Lemma}
\newtheorem*{fact}{Fact}
\newtheorem*{idea}{Idea}
\newtheorem*{clm}{Claim}
\newtheorem*{rmk}{Remark}
\usepackage[ruled]{algorithm2e}

\usepackage{ulem}
\makeatletter

\def\lf{\left\lfloor}   
\def\rf{\right\rfloor}
\def\lc{\left\lceil}   
\def\rc{\right\rceil}
\def\st{\text{ s.t. }}
\def\1{^{-1}}
\def\ind{\mathbf{1}}
\def\R{\mathbb{R}}
\def\Q{\mathbb{Q}}
\def\Z{\mathbb{Z}}
\def\C{\mathbb{C}}
\def\I{\mathbb{I}}
\def\N{\mathbb{N}}
\def\F{\mathbb{F}}
\def\A{\mathbb{A}}
\def\Li{\text{Li}}
\def\th{^\text{th}}
\def\sp{\text{Sp}}
\def\opn{\left\{}
\def\cls{\right\}}
\def\Aut{\text{Aut}}
\def\PG{\text{PG}}
\def\GL{\text{GL}}
\def\PGL{\text{PGL}}
\def\Cov{\text{Cov}}
\def\Pack{\text{Pack}}
\def\PgamL{\text{P}\Gamma\text{L}}
\def\gamL{\Gamma\text{L}}
\def\cl{\text{cl}}
\def\stbar{\ \middle\vert\ }
\def\partdone{\hphantom{1} \hfill \(\triangle\)}
\def\s0{_0}
\def\s1{_1}
\def\s2{_2}
\def\id{\mathrm{id}}
\def\topn{\text{ open}}
\def\Bd{\text{Bd }}
\renewcommand{\P}{\mathbb{P}}
\newcommand{\leg}[2]{\left( \frac{#1}{#2} \right)}
\renewcommand*\env@matrix[1][*\c@MaxMatrixCols c]{%
   \hskip -\arraycolsep
   \let\@ifnextchar\new@ifnextchar
   \array{#1}}
\makeatother

% These two lines suppress the warning generated 
% by amsmath for overwriting the choose command  
% because it's annoying. This probably has unint-
% ended ramifications somewhere else, but I'm too
% lazy to actually figure that out, so we'll cro-
% ss that bridge when we come to it lol.
\renewcommand{\choose}[2]{\left( {#1 \atop #2} \right)}
\WarningFilter{amsmath}{Foreign command} 

\renewcommand{\mod}[1]{\ (\mathrm{mod}\ #1)}
\renewcommand{\vec}[1]{\mathbf{#1}}

\let\oldprime\prime
\def\prime{^\oldprime}

\usepackage{float}
\restylefloat{figure}

\usepackage{cleveref}
\Crefname{thm}{Theorem}{Theorems}

% Comment commands for co-authors
\newcommand{\kmd}[1]{{\color{purple} #1}}

\newcolumntype{L}{>{$}l<{$}}
% Bib matter
\let\oldepsilon\epsilon
\def\epsilon{\varepsilon}

\let\oldphi\phi
\def\phi{\varphi}

%%% Local Variables:
%%% mode: plain-tex
%%% TeX-master: t
%%% End:

\graphicspath{{./img/}}

\begin{document}
\noindent \textbf{Math 171} \hfill \textbf{Professor Sebastian Bozlee} \\
\textbf{Scribed by: Kyle Dituro} \hfill \textbf{April 26, 2023}\hrule
\vspace{.2in}
We will pick up with the tube lemma that we left unproven at the end of the last class.
\begin{lm}[Tube Lemma]
  If \(X\) is a topological space and \(Y\) is a compact space. Let \(x_0 \in X\), and let \(N \subseteq X \times Y\) be an open subset containing \(x_0 \times Y\), then \(N\) contains some ``tube'' \(W \times Y\), where \(W\) is an open neighborhood of \(x_0\) in \(X\).
\end{lm}
\begin{proof}
  For each point \((x_0, y) \in x_0 \times Y \subseteq N\). choose a basic open subset \(U_y \times V_y\) where \(U_y \subseteq X\) is open, \(V_y \subseteq Y\) open, and \((x_0, y) \in U_y \times V_y \subseteq N\). Then \(\opn U_y \times V_y \cls_{y \in Y}\) is an open covering of \(x_0 \times Y\). But \(x_0 \times Y \cong Y\), so \(x_0 \times Y\) is compact, so there exists a \(y_1, \ldots, y_n \in Y\) such that \(Y \subseteq (U_{y_1} \times V_{y_1})\cup \ldots \cup (U_{y_n} \times V_{y_n})\). Let \(W = U_{y_1} \cap \ldots \cap U_{y_n}\). Then this is an open neighborhood of \(x_0 \in X\). Moreover,

  \begin{align*}
    W \times Y &= W \times \left(\bigcup_{i=1}^nV_{y_n}\right) \\
               &=\bigcup_{i = 1}^n (W \times V_{y_n}) \\
               &\subseteq \bigcup_{i = 1}^n ( U_{y_i} \times V_{y_i}) \subseteq N
  \end{align*}
\end{proof}

\begin{thm}
  If \(X\) and \(Y\) are compact topological spaces, then so is \(X \times Y\).
\end{thm}

\begin{proof}
  Let \(\opn A \cls_{i \in I}\) be an open cover of \(X \times Y\). Then, as in the proof of the tube lemma, \(x_0 \times Y\) is compact, so finitely many of the \(A_i\)s cover \(x_0 \times Y\). Say \(A_{i_1} \cup \ldots \cup A_{i_n} \supseteq x_0 \times Y\). Let \(N = A_{i_1} \cup \ldots \cup A_{i_m}\).

  Now by the tube lemma, there is an open neighborhood \(W_{x_0}\) of \(x_0\) in \(X\) such that \(W_{x_0} \times Y\subseteq N = A_{i_1} \cup \ldots\cup A_{i_n}\)

  Observe that  \(\opn W_x \cls_{x \in X}\) is an open cover of \(X\). Since \(X\) is compact, there exists \(x_1, \ldots, x_n\) such that \(W_{x_1} \cup \ldots \cup W_{x_n} = X\) then \((W_{x_1} \times Y) \cup (W_{x_2} \times Y) \cup \ldots \cup (W_{x_n} \times Y) = X \times Y\).

  Since there are finitely many \(W_{x_i}\)s, and each of the \(W_{x_i} \times Y\) admits a cover by finitely many elements of \(\opn A_i \cls_{i \in I}\), \(X \times Y\) also admits a finite cover by \(A_i\)s. Therefore \(X \times Y\) is compact.
\end{proof}

Now moving on to a discussion of compact subspaces of \(\R^n\).

We begin with the long-alluded to proof that closed intervals are compact.
\begin{thm}
  A finite closed interval \([a, b] \subseteq \R\) is compact
\end{thm}
\begin{proof}
  Let \(\mathcal{U} = \opn U_i \cls_{i \in I}\) be an open cover \([a, b]\) in the subspace topology.

  \begin{enumerate}
  \item [\emph{Step 1:}] Our goal is to show that if \(x\) is a point of \([a, b)\) then there is \(y > x\) in \([a, b]\) such that the interval \([x, y]\) is covered by just one element of \(\mathcal{U}\).

    Let \(x \in [a, b)\). Since \(\mathcal{U}\) is a cover, there is an element \(U \in \mathcal{U}\) such that \(x \in U\). Since \(U\) is open and \(x \neq b\), there is some \(c\) in \((x, b)\) such that \([x, c) \subseteq U\). Chose \(y \in (x, c)\). Then \([x, y] \subseteq U\) \partdone
  \item [\emph{Step 2:}] Let \(C\) be the set of all points \(y > a\) of \([a, b]\) such that the interval \([a, y]\) can be covered by finitely many elements of \(\mathcal{U}\). 

    Applying \emph{Step 1} to \(x = a\) we get that at least one \(y \in C\), so \(C\) is nonempty. \(C\) is bounded above by \(b\), so there exists a supremum \(c\) of \(C\) \partdone
  \item [\emph{Step 3:}] We now want to show \(c \in C\).

    Suppose not. Choose an element \(U \in \mathcal{U}\) containing \(c\). We know that \(c > a\), so there exists \(d \in [a, b]\) such that \((d, c] \subseteq U\). Choose an element \(z\) of \((d, c]\). We must have \(z \in C\) since \(c\) is least upper bound. \partdone

    Then since \(z \in C\), the interval \([a, z]\) can be covered by finitely many elements of \(\mathcal{U}\), say \(U_1, \ldots, U_n\). But then \(U_1 \cup \ldots \cup U_n \cup U \supseteq [a, z] \cup [z, c] = [a, c]\). Then \(c \in C\) \nope, so \(c \in C\). \partdone
  \item [\emph{Step 4:}] Now we show that \(c = b\).

    Once again, suppose not. Then \(c < b\), so by \emph{Step 1}, there is a \(y \in (c, b]\) and \(U \in \mathcal{U}\) such that \([c, y] \subseteq U\). By definition of \(C\), since \(c \in C\) there is a finite collection of elements of \(\mathcal{U}\), say \(U_1, \ldots, U_n\) covering \([a, c]\). Then \[U_1 \cup \ldots \cup U_n \cup U \supseteq [a, c] \cup [c, y] = [a, y].\]

    But then \(y > c\) and \(y \in C\), which contradicts that \(c\) is an upper bound on \(C\) \nope. \partdone
  \end{enumerate}

  Therefore \([a, b]\) is compact.
\end{proof}

\begin{df}
  A subset \(A \subseteq \R^n\) is said to be \textbf{bounded} if there is a radius \(r\) such that \(A \subseteq B(\vec{0}, r)\). (where \(B(\vec{x}, r)\) is a ball centered at \(\vec{x}\) with radius \(r\)). 
\end{df}

\begin{fact}
  \(A\) is bounded iff there exists \(b \in \R\) such that \(A \subseteq [-b, b]^n\). 
\end{fact}

\begin{thm}[Extreme Value Theorem]
  Let \(X\) be a compact topological space and \(f: X \to \R\) be a continuous function. Then there are elements \(a, b \in X\) such that \(f(a) \leq f(x) \leq f(b)\) for all \(x \in X\). 
\end{thm}
\end{document}
%%% Local Variables:
%%% mode: latex
%%% TeX-master: t
%%% End:

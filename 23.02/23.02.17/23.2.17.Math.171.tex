\documentclass[12pt, twosided]{article}

\usepackage[letterpaper,bindingoffset=0in,%
            left=1in,right=1in,top=1in,bottom=1in,%
            footskip=.25in]{geometry}

\usepackage{mathtools}
\usepackage{graphicx}

\usepackage{setspace}
\setstretch{1.1}

\usepackage{amsmath}
\usepackage{amsfonts}
\usepackage{amsthm}
\usepackage{amssymb}
\usepackage{csquotes}
\usepackage{relsize}

\usepackage{tikz}
\usetikzlibrary{cd}
\usetikzlibrary{fit,shapes.geometric}
\tikzset{%  
    mdot/.style={draw, circle, fill=black},
    mset/.style={draw, ellipse, very thick},
}

\usepackage{hhline}
\usepackage{systeme}
\usepackage{mathrsfs}
\usepackage{hyperref}
\usepackage{mathtools}  
\usepackage{silence}
\usepackage{blkarray}
\usepackage{float}
\usepackage{framed}
\usepackage{array}
\usepackage{stmaryrd}
\usepackage{extarrows}
\usepackage{caption}
\captionsetup[figure]{labelfont={bf},name={Fig.},labelsep=period}

\theoremstyle{definition}
\newtheorem{df}{Definition}
\newtheorem{exa}{Example}
\newtheorem{ques}{Question}
\newtheorem{exr}{Exercise}
\newtheorem*{note}{Note}
\theoremstyle{plain}
\newtheorem{thm}{Theorem}
\newtheorem{prop}{Proposition}
\newtheorem{conj}{Conjecture}
\newtheorem{cor}{Corollary}
\newtheorem{lm}{Lemma}
\newtheorem*{fact}{Fact}
\newtheorem*{idea}{Idea}
\newtheorem*{clm}{Claim}
\newtheorem*{rmk}{Remark}
\usepackage[ruled]{algorithm2e}

\usepackage{ulem}
\makeatletter

\def\lf{\left\lfloor}   
\def\rf{\right\rfloor}
\def\lc{\left\lceil}   
\def\rc{\right\rceil}
\def\st{\text{ s.t. }}
\def\1{^{-1}}
\def\ind{\mathbf{1}}
\def\R{\mathbb{R}}
\def\Q{\mathbb{Q}}
\def\Z{\mathbb{Z}}
\def\C{\mathbb{C}}
\def\I{\mathbb{I}}
\def\N{\mathbb{N}}
\def\F{\mathbb{F}}
\def\A{\mathbb{A}}
\def\Li{\text{Li}}
\def\th{^\text{th}}
\def\sp{\text{Sp}}
\def\opn{\left\{}
\def\cls{\right\}}
\def\Aut{\text{Aut}}
\def\PG{\text{PG}}
\def\GL{\text{GL}}
\def\PGL{\text{PGL}}
\def\Cov{\text{Cov}}
\def\Pack{\text{Pack}}
\def\PgamL{\text{P}\Gamma\text{L}}
\def\gamL{\Gamma\text{L}}
\def\cl{\text{cl}}
\def\stbar{\ \middle\vert\ }
\def\partdone{\hphantom{1} \hfill \(\triangle\)}
\def\s0{_0}
\def\s1{_1}
\def\s2{_2}
\def\id{\mathrm{id}}
\def\topn{\text{ open}}
\def\Bd{\text{Bd }}
\renewcommand{\P}{\mathbb{P}}
\newcommand{\leg}[2]{\left( \frac{#1}{#2} \right)}
\renewcommand*\env@matrix[1][*\c@MaxMatrixCols c]{%
   \hskip -\arraycolsep
   \let\@ifnextchar\new@ifnextchar
   \array{#1}}
\makeatother

% These two lines suppress the warning generated 
% by amsmath for overwriting the choose command  
% because it's annoying. This probably has unint-
% ended ramifications somewhere else, but I'm too
% lazy to actually figure that out, so we'll cro-
% ss that bridge when we come to it lol.
\renewcommand{\choose}[2]{\left( {#1 \atop #2} \right)}
\WarningFilter{amsmath}{Foreign command} 

\renewcommand{\mod}[1]{\ (\mathrm{mod}\ #1)}
\renewcommand{\vec}[1]{\mathbf{#1}}

\let\oldprime\prime
\def\prime{^\oldprime}

\usepackage{float}
\restylefloat{figure}

\usepackage{cleveref}
\Crefname{thm}{Theorem}{Theorems}

% Comment commands for co-authors
\newcommand{\kmd}[1]{{\color{purple} #1}}

\newcolumntype{L}{>{$}l<{$}}
% Bib matter
\let\oldepsilon\epsilon
\def\epsilon{\varepsilon}

\let\oldphi\phi
\def\phi{\varphi}

%%% Local Variables:
%%% mode: plain-tex
%%% TeX-master: t
%%% End:

\graphicspath{{./img/}}

\begin{document}
\noindent \textbf{Math 171} \hfill \textbf{Professor Sebastian Bozlee} \\
\textbf{Scribed by: Kyle Dituro} \hfill \textbf{February 17, 2023}\hrule
\vspace{.2in}
We begin with a reminder of what will be one of the most fundimental definitions for this course.

\begin{df}
  Let \(X\) be a set. A \textbf{topology on} \(X\) is a set \(\tau\) of subsets 
of \(X\) such that:
\begin{enumerate}
  \item \(\emptyset, X \in \tau\)
  \item If \(U_i \in \tau\) of each \(i \in I\) then \(\bigcup_{i \in I} U_i \in \tau\)
  \item If \(U_1, \ldots, U_n \in \tau\), then \(\bigcap_{i = 1}^n U_i \in \tau\).
  \end{enumerate}
    A \textbf{topological space} is a pair \((X, \tau)\) where \(X\) is a set and \(\tau\) is a topology on \(X\). Then the elements of \(\tau\) are called the \textbf{open subsets} of \(X\).

\end{df}

And we remember the definition of continuity:

\begin{df}
  Let \((X, \tau_X), (Y, \tau_Y)\) be topological spaces. A \textbf{continuous function} \(f: (X, \tau_X) \to (Y, \tau_Y)\) is a function \(f: X \to Y\) such that whenever \(U \in \tau_Y\), we have that \(f\1(U) \in \tau_X\).
\end{df}

And we'll proceed with a number of examples to guide our intuition of open-ness in a topological sense and continuity.

\begin{exa}
  Let \(X = \mathbb{R}\) be given. We know that the usual notion of open sets defines a topology on \(X\):
  \begin{align*}
    \tau_1 &= \opn \text{open subsets of } \R \cls
  \end{align*}

  However, this is not the only topology that we can endow the real numbers with. In fact, I claim that: \[\tau_2 = \tau_1 \cup \opn \opn 0 \cls \cup U\ \middle\vert U \in \tau_1 \cls\]
  Let's check:
  \begin{itemize}
  \item \(\emptyset, \R \in \tau_2\)
  \item \(\opn U_i \cls_{i \in I}\)with \(U_i \in \tau_1\), \(\{\{0\} \cup U_j\}_{j \in J}\) where \(U)j \in \tau_1\). The union of these is either \(\bigcup_{i \in I}U_i\) if \(J\) is empty or \(\{0\} \cup \left( \bigcup_{i \in I \cup J} U_i \right)\)
  \item \(U_1, \ldots, U_n, \{0\} \cup U_{n+1}, \ldots, \{0\} \cup U_{n + m}\), the intersection of these tings is either \(\{0\} \cup \left( \bigcap U_i\right)\)if \(n = 0\) or \(\bigcap U_i\) if \(n \neq 0\).
  \end{itemize}
      
\end{exa}

\begin{exa}
  Consider the function
  \(f(x) = \begin{cases} 1 & x = 0 \\ 0 & x \neq 0 \end{cases}\)
  Under the topology \(\tau_2\), this function is actually continuous.
\end{exa}

\begin{fact}
  The continuous funcions \(g: (\R, \tau_2) \to (\R, \tau_1)\)  are the funtions continuous at all \(x \in \R - \{0\}\) in the analysis sense of continuity.
  The continuous functions \(f:(\R, \tau_1) \to (\R, \tau_2)\) are the continuous function with \(g(\R) \subseteq (-\infty, 0), (0, \infty),\) or \(\{0\}\).
\end{fact}

\begin{idea}
  Let \(f: X \to Y\), the more open subsets \(Y\) has, the harder it is for \(f\) to be continuous, and likewise the more open susbsets \(X\) has, the easier it s.
\end{idea}


\begin{df}
  Let \(X\) be a set, \(\tau_1, \tau_2\) two topologies on \(X\). We say that \(\tau_1\) is coarser than \(\tau_2\) and that \(\tau_2\) is finer than \(\tau_1\) if \(\tau_1 \subseteq \tau_2\).
\end{df}

\begin{exa}
  Notice that the discrete topology is always the finest topology on a set \(X\), and that the indiscrete topology is the finest.
\end{exa}

\begin{prop}
  Let \(f: X \to Y\) be a function between topological spaces.
  \begin{enumerate}
  \item If \(X\) has the discrete topology then \(f\) is continuous.
  \item If \(Y\) has the indiscrete topology then \(f\) is continuous.
  \end{enumerate}
\end{prop}
\begin{proof}
  \begin{enumerate}
  \item let \(f: X \to Y\) be any function. Let \(U \subseteq Y\) be an open subset. This is obvious because every subset of \(X\) is open, so of course \(f\1(U)\) is open.
  \item Let \(f:X \to Y\) be any function. Let \(U \subseteq Y\) be an open set. Now since \(Y\) has the indiscrete topology, we need only check \(f\1(Y)\) and \(f\1(\emptyset)\) but \(f\1(Y) = X\) and \(f\1(\emptyset) = \emptyset\).
  \end{enumerate}
  Thus, our function is continuous in either case.
\end{proof}

\begin{prop}
  \(\mathrm{id}_X: (X, \tau_2)) \to (X, \tau_1)\)is continuous iff \(\tau_2\) is finer than \(\tau_1\).
  In particular \(\mathrm{id}_X: (X, \tau) \to (X, \tau)\) is continuous.
\end{prop}
\begin{proof}
  \(\mathrm{id}_X:(X, \tau_2) \to (X, \tau_1)\) is continuous iff \(\forall U \in \tau_1\), \(\mathrm{id}\1_X(U) \in \tau_2\) iff \(\forall U \in \tau_1\), have \(U \in tau_2\) iff \(\tau_1 \subseteq \tau_2\) which is the definition of \(\tau_2\) is finer than \(\tau_1\).
\end{proof}

\begin{lm}
  Let \(f: X \to Y\), \(g: Y \to Z\) be functions, \(W \subseteq Z\). Then \((g \circ f)\1(W) = f\1\left(g\1\left(W\right)\right)\).
\end{lm}
\begin{proof}
  \begin{align*}
    f\1(g\1(W)) &= \opn x \in X \middle\vert f(x) \in g\1(W) \cls \\
                &= \opn x \in X \middle\vert f(x) \in \opn y \in Y \middle \vert g(y) \in W \cls \cls \\
                &= \opn x \in X \middle\vert g(f(x)) \in W \cls \\
                &= (g \circ f)\1 (W)
  \end{align*}
\end{proof}

\begin{prop}
  Let \(f: X \to Y\), \(g: Y \to Z\) be continuous. Then \((g \circ f): X \to Z\) is as well.
\end{prop}
\begin{proof}
  Let \(W \subseteq Z\) be open. Then
  \begin{align*}
    (g \circ f)\1(W) &= f\1(g\1(W))
  \end{align*}
  Notice here that we are taking the pre-image of an open set by continuous functions twice. Thus, \(g \circ f\) is continuous.
\end{proof}
And so it was that Kyle never wrote an \(\epsilon\)-\(\delta\) proof again. And we have officielly moved out of the category of Sets, and have moved into the category of Topologies.
\end{document}
%%% Local Variables:
%%% mode: latex
%%% TeX-master: t
%%% End:

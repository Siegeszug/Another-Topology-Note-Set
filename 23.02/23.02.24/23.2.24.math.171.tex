\documentclass[12pt, twosided]{article}

\usepackage[letterpaper,bindingoffset=0in,%
            left=1in,right=1in,top=1in,bottom=1in,%
            footskip=.25in]{geometry}

\usepackage{mathtools}
\usepackage{graphicx}

\usepackage{setspace}
\setstretch{1.1}

\usepackage{amsmath}
\usepackage{amsfonts}
\usepackage{amsthm}
\usepackage{amssymb}
\usepackage{csquotes}
\usepackage{relsize}

\usepackage{tikz}
\usetikzlibrary{cd}
\usetikzlibrary{fit,shapes.geometric}
\tikzset{%  
    mdot/.style={draw, circle, fill=black},
    mset/.style={draw, ellipse, very thick},
}

\usepackage{hhline}
\usepackage{systeme}
\usepackage{mathrsfs}
\usepackage{hyperref}
\usepackage{mathtools}  
\usepackage{silence}
\usepackage{blkarray}
\usepackage{float}
\usepackage{framed}
\usepackage{array}
\usepackage{stmaryrd}
\usepackage{extarrows}
\usepackage{caption}
\captionsetup[figure]{labelfont={bf},name={Fig.},labelsep=period}

\theoremstyle{definition}
\newtheorem{df}{Definition}
\newtheorem{exa}{Example}
\newtheorem{ques}{Question}
\newtheorem{exr}{Exercise}
\newtheorem*{note}{Note}
\theoremstyle{plain}
\newtheorem{thm}{Theorem}
\newtheorem{prop}{Proposition}
\newtheorem{conj}{Conjecture}
\newtheorem{cor}{Corollary}
\newtheorem{lm}{Lemma}
\newtheorem*{fact}{Fact}
\newtheorem*{idea}{Idea}
\newtheorem*{clm}{Claim}
\newtheorem*{rmk}{Remark}
\usepackage[ruled]{algorithm2e}

\usepackage{ulem}
\makeatletter

\def\lf{\left\lfloor}   
\def\rf{\right\rfloor}
\def\lc{\left\lceil}   
\def\rc{\right\rceil}
\def\st{\text{ s.t. }}
\def\1{^{-1}}
\def\ind{\mathbf{1}}
\def\R{\mathbb{R}}
\def\Q{\mathbb{Q}}
\def\Z{\mathbb{Z}}
\def\C{\mathbb{C}}
\def\I{\mathbb{I}}
\def\N{\mathbb{N}}
\def\F{\mathbb{F}}
\def\A{\mathbb{A}}
\def\Li{\text{Li}}
\def\th{^\text{th}}
\def\sp{\text{Sp}}
\def\opn{\left\{}
\def\cls{\right\}}
\def\Aut{\text{Aut}}
\def\PG{\text{PG}}
\def\GL{\text{GL}}
\def\PGL{\text{PGL}}
\def\Cov{\text{Cov}}
\def\Pack{\text{Pack}}
\def\PgamL{\text{P}\Gamma\text{L}}
\def\gamL{\Gamma\text{L}}
\def\cl{\text{cl}}
\def\stbar{\ \middle\vert\ }
\def\partdone{\hphantom{1} \hfill \(\triangle\)}
\def\s0{_0}
\def\s1{_1}
\def\s2{_2}
\def\id{\mathrm{id}}
\def\topn{\text{ open}}
\def\Bd{\text{Bd }}
\renewcommand{\P}{\mathbb{P}}
\newcommand{\leg}[2]{\left( \frac{#1}{#2} \right)}
\renewcommand*\env@matrix[1][*\c@MaxMatrixCols c]{%
   \hskip -\arraycolsep
   \let\@ifnextchar\new@ifnextchar
   \array{#1}}
\makeatother

% These two lines suppress the warning generated 
% by amsmath for overwriting the choose command  
% because it's annoying. This probably has unint-
% ended ramifications somewhere else, but I'm too
% lazy to actually figure that out, so we'll cro-
% ss that bridge when we come to it lol.
\renewcommand{\choose}[2]{\left( {#1 \atop #2} \right)}
\WarningFilter{amsmath}{Foreign command} 

\renewcommand{\mod}[1]{\ (\mathrm{mod}\ #1)}
\renewcommand{\vec}[1]{\mathbf{#1}}

\let\oldprime\prime
\def\prime{^\oldprime}

\usepackage{float}
\restylefloat{figure}

\usepackage{cleveref}
\Crefname{thm}{Theorem}{Theorems}

% Comment commands for co-authors
\newcommand{\kmd}[1]{{\color{purple} #1}}

\newcolumntype{L}{>{$}l<{$}}
% Bib matter
\let\oldepsilon\epsilon
\def\epsilon{\varepsilon}

\let\oldphi\phi
\def\phi{\varphi}

%%% Local Variables:
%%% mode: plain-tex
%%% TeX-master: t
%%% End:

\graphicspath{{./img/}}

\begin{document}
\noindent \textbf{Math 171} \hfill \textbf{Professor Sebastian Bozlee} \\
\textbf{Scribed by: Kyle Dituro} \hfill \textbf{February 24, 2023}\hrule
\vspace{.2in}

Recall the definition of the subspace topology:

\begin{df}
  If \((X, \tau_X)\) is a topological space, and \(Y \subseteq X\), then the \textbf{subspace topology on }\(Y\) is the topology \[\tau_{\text{sub}} = \opn U \cap Y \middle\vert U \in \tau_X \cls\].
\end{df}

We now give a proposition which somehow justifies that this is not just any topology, but is somehow a good choice.

\begin{prop}
  With notation as above:
  \begin{enumerate}
  \item \(i: (Y, \tau_{\mathrm{sub}}) \to (X, \tau_X)\) taking \(y \mapsto y\) (the inclusion function) is continuous.
  \item \(\tau_{\mathrm{sub}}\) is the coarsest topology on \(Y\) such that \(i\) is continuous.
  \item If \(f: (X, \tau_X) \to (Z, \tau_Z)\) \(x \mapsto f(x)\) is continuous, then \(f\vert_y : (Y, \tau_\mathrm{sub}) \to (Z, \tau_Z)\) \(y \mapsto f(y)\) is alco continuous.
  \item If \((W, \tau_W)\) is a topological space and \(g: W \to Y\) is a function, then
    \begin{align*}
      g:(W, \tau_W) \to (Y, \tau_{\mathrm{sub}}) \text{ is continuous } &\Leftrightarrow& i \circ g : (W , \tau_W) \to (X, \tau_X) \text{ is continuous} \\
      w \mapsto g(w) & & w \mapsto g(w)
    \end{align*}
  \end{enumerate}
\end{prop}
\begin{proof}
  \begin{enumerate}
  \item This part is mosly trivial (try writing out what \(i\1\) of an open set must be! The proof follows directly).
  \item Let \(\tau_2\) be a topology on \(Y\) such that \(i: (Y, \tau_2) \to (X, \tau_X)\) is continuous.
    Since \(i\) is continuous, for all \(U \subseteq X\) an open subset of \(X\), we have that \(i\1(U): U \cap Y\) is open in \(Y\) with topology \(\tau_2\). This shows that for any \(U \subseteq X\) an open subset of \(X\), \(U \cap Y \in \tau_2\) i.e. an arbitrary element of \(\tau_\mathrm{sub}\) also belongs to \(\tau_2\). So \(\tau_2 \supseteq \tau_{\mathrm{sub}}\), and, by definition \(\tau_{\mathrm{sub}}\) is coarser than \(\tau_2\).
  \item \[(Y, \tau_{\mathrm{sub}}) \xrightarrow[\mathrm{cont}]{i} (X, \tau_X) \xrightarrow[\mathrm{cont}]{f} (Z, \tau_Z)\] is a composition of continuous functions which is continuous, so \(f\vert_y = f \circ i\) is continuous.
  \item
    \begin{itemize}
    \item [\((\Rightarrow)\)] if \(g: (W, \tau_W) \to (Y, \tau_{\mathrm{sub}})\) is continuous, then again \(i \circ g\) is a ocmposite of continuous functions, so \(i \circ g\) is continuous.
    \item [\((\Leftarrow)\)] Assume \(i \circ g\) is continuous. Let some \(U \subseteq Y\) is open. By definition \(U = V \cap Y\) for some \(V \subseteq X\) an open subset of \(X\). So we know \((i \circ g)\1 (V)\) is open in \(W\).
      \begin{align*}
        (i \circ g)\1(V) &= g\1(i\1(V)) \\
                         &= g\1(V \cap Y) \\
                         &= g\1(U).
      \end{align*}
      So \(g\1(U)\) is open, and thus \(g\) is continuous.
    \end{itemize}
  \end{enumerate}
\end{proof}

Notice that if \(U\) is open in \(Y\), we may not have that \(U\) is open in \(X\). \( \opn 0, 1 \cls \subseteq \R\), then \(\opn 0, 1 \cls\) is an open subset of \(Y\), but not an open subset of \(X\).

\begin{prop}
  If \(Y\) is an open subset of a topological space \(X\) and \(U \subseteq Y\), then \(U \subseteq Y\) is and open subset of \(Y\) in the subspace topology iff \(U\) is open in \(X\).
\end{prop}

\begin{proof}
  \begin{enumerate}
  \item [\((\Rightarrow)\)] Let \(U\) be an open subset of \(Y\) in the subspace topology. Then there exists some open set \(V \subseteq X\) such that \(V \cap Y = U\). But we know that both \(V\) and \(Y\) are open in \(X\), their intersection must also be open in \(X\).
  \item [\((\Leftarrow)\)] Let \(U \subseteq Y\), and we know that \(U\) is open in \(X\). But since \(U \subseteq Y\), \(U = U \cap Y\), but then since \(U\) is open in \(X\) by assumption, \(U\) is open in \(Y\).
  \end{enumerate}
\end{proof}

Remember that in the early problem, we (should have) realized that \[\opn (a, b)\ \middle\vert\ a < b\cls \cup \opn \emptyset, \R \cls\] is not a topology.
\begin{exr}
  Why?
\end{exr}

Notice that if we allow also for unions of these open intervals, we realize that we get the usual topology on \(\R\), which better be a topology (otherwise our definition would be quite bad).

\begin{df}
  let \(X\) be a set. We say that a collection \(\mathcal{B}\) of subsets of \(X\) is a \textbf{basis} for a topology on \(X\) if:
  \begin{enumerate}
  \item For each point \(x\in X\), there is a \textbf{basis element} (or \textbf{basic open subset}) \(B \in \mathcal{B}\) such that \(x \in B\)
  \item For each pair of basis elements \(B_1, B_2 \in \mathcal{B}\) and \(x \in B_1 \cap B_2\) there exists a basis element \(B_3 \in \mathcal{B}\) such that \(x \in B_3 \subseteq B_1 \cap B_2\).
  \end{enumerate}

  The \textbf{topology} \(\tau\) \textbf{generated by the basis} \(\mathcal{B}\) is the topology on \(X\) where:
  \begin{enumerate}
  \item a subset \(U \subseteq X\) is said to be open if for each \(x \in U\) there is a basis element \(B \in \mathcal{B}\) such that \(x \in B\) and \(B \subseteq U\).
  \end{enumerate}
\end{df}

\begin{exa}
  In the usual topology on \(\R^n\), the open balls \(B(\vec{x}, \epsilon)\) form a basis for the usual topology.
\end{exa}
\end{document}
%%% Local Variables:
%%% mode: latex
%%% TeX-master: t
%%% End:

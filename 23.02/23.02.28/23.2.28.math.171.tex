\documentclass[12pt, twosided]{article}

\usepackage[letterpaper,bindingoffset=0in,%
            left=1in,right=1in,top=1in,bottom=1in,%
            footskip=.25in]{geometry}

\usepackage{mathtools}
\usepackage{graphicx}

\usepackage{setspace}
\setstretch{1.1}

\usepackage{amsmath}
\usepackage{amsfonts}
\usepackage{amsthm}
\usepackage{amssymb}
\usepackage{csquotes}
\usepackage{relsize}

\usepackage{tikz}
\usetikzlibrary{cd}
\usetikzlibrary{fit,shapes.geometric}
\tikzset{%  
    mdot/.style={draw, circle, fill=black},
    mset/.style={draw, ellipse, very thick},
}

\usepackage{hhline}
\usepackage{systeme}
\usepackage{mathrsfs}
\usepackage{hyperref}
\usepackage{mathtools}  
\usepackage{silence}
\usepackage{blkarray}
\usepackage{float}
\usepackage{framed}
\usepackage{array}
\usepackage{stmaryrd}
\usepackage{extarrows}
\usepackage{caption}
\captionsetup[figure]{labelfont={bf},name={Fig.},labelsep=period}

\theoremstyle{definition}
\newtheorem{df}{Definition}
\newtheorem{exa}{Example}
\newtheorem{ques}{Question}
\newtheorem{exr}{Exercise}
\newtheorem*{note}{Note}
\theoremstyle{plain}
\newtheorem{thm}{Theorem}
\newtheorem{prop}{Proposition}
\newtheorem{conj}{Conjecture}
\newtheorem{cor}{Corollary}
\newtheorem{lm}{Lemma}
\newtheorem*{fact}{Fact}
\newtheorem*{idea}{Idea}
\newtheorem*{clm}{Claim}
\newtheorem*{rmk}{Remark}
\usepackage[ruled]{algorithm2e}

\usepackage{ulem}
\makeatletter

\def\lf{\left\lfloor}   
\def\rf{\right\rfloor}
\def\lc{\left\lceil}   
\def\rc{\right\rceil}
\def\st{\text{ s.t. }}
\def\1{^{-1}}
\def\ind{\mathbf{1}}
\def\R{\mathbb{R}}
\def\Q{\mathbb{Q}}
\def\Z{\mathbb{Z}}
\def\C{\mathbb{C}}
\def\I{\mathbb{I}}
\def\N{\mathbb{N}}
\def\F{\mathbb{F}}
\def\A{\mathbb{A}}
\def\Li{\text{Li}}
\def\th{^\text{th}}
\def\sp{\text{Sp}}
\def\opn{\left\{}
\def\cls{\right\}}
\def\Aut{\text{Aut}}
\def\PG{\text{PG}}
\def\GL{\text{GL}}
\def\PGL{\text{PGL}}
\def\Cov{\text{Cov}}
\def\Pack{\text{Pack}}
\def\PgamL{\text{P}\Gamma\text{L}}
\def\gamL{\Gamma\text{L}}
\def\cl{\text{cl}}
\def\stbar{\ \middle\vert\ }
\def\partdone{\hphantom{1} \hfill \(\triangle\)}
\def\s0{_0}
\def\s1{_1}
\def\s2{_2}
\def\id{\mathrm{id}}
\def\topn{\text{ open}}
\def\Bd{\text{Bd }}
\renewcommand{\P}{\mathbb{P}}
\newcommand{\leg}[2]{\left( \frac{#1}{#2} \right)}
\renewcommand*\env@matrix[1][*\c@MaxMatrixCols c]{%
   \hskip -\arraycolsep
   \let\@ifnextchar\new@ifnextchar
   \array{#1}}
\makeatother

% These two lines suppress the warning generated 
% by amsmath for overwriting the choose command  
% because it's annoying. This probably has unint-
% ended ramifications somewhere else, but I'm too
% lazy to actually figure that out, so we'll cro-
% ss that bridge when we come to it lol.
\renewcommand{\choose}[2]{\left( {#1 \atop #2} \right)}
\WarningFilter{amsmath}{Foreign command} 

\renewcommand{\mod}[1]{\ (\mathrm{mod}\ #1)}
\renewcommand{\vec}[1]{\mathbf{#1}}

\let\oldprime\prime
\def\prime{^\oldprime}

\usepackage{float}
\restylefloat{figure}

\usepackage{cleveref}
\Crefname{thm}{Theorem}{Theorems}

% Comment commands for co-authors
\newcommand{\kmd}[1]{{\color{purple} #1}}

\newcolumntype{L}{>{$}l<{$}}
% Bib matter
\let\oldepsilon\epsilon
\def\epsilon{\varepsilon}

\let\oldphi\phi
\def\phi{\varphi}

%%% Local Variables:
%%% mode: plain-tex
%%% TeX-master: t
%%% End:

\graphicspath{{./img/}}

\begin{document}
\noindent \textbf{Math 171} \hfill \textbf{Professor Sebastian Bozlee} \\
\textbf{Scribed by: Kyle Dituro} \hfill \textbf{February 28, 2023}\hrule
\vspace{.2in}

We begin by recalling the definition of a basis.

\begin{df}
  let \(X\) be a set. We say that a collection \(\mathcal{B}\) of subsets of \(X\) is a \textbf{basis} for a topology on \(X\) if:
  \begin{enumerate}
  \item For each point \(x\in X\), there is a \textbf{basis element} (or \textbf{basic open subset}) \(B \in \mathcal{B}\) such that \(x \in B\)
  \item For each pair of basis elements \(B_1, B_2 \in \mathcal{B}\) and \(x \in B_1 \cap B_2\) there exists a basis element \(B_3 \in \mathcal{B}\) such that \(x \in B_3 \subseteq B_1 \cap B_2\).
  \end{enumerate}

  The \textbf{topology} \(\tau\) \textbf{generated by the basis} \(\mathcal{B}\) is the topology on \(X\) where:
  \begin{enumerate}
  \item a subset \(U \subseteq X\) is said to be open if for each \(x \in U\) there is a basis element \(B \in \mathcal{B}\) such that \(x \in B\) and \(B \subseteq U\).
  \end{enumerate}
\end{df}

\begin{exa}
  In the usual topology on \(\R^n\), the open balls \(B(\vec{x}, \epsilon)\) form a basis for the usual topology. I.E. the set
  \[\mathcal{B} = \opn B(\vec{x}, \epsilon) \middle\vert \vec{x} \in \R^n, \epsilon > 0 \cls\] Is a basis for the usual topology on \(\R^n\).

  \begin{proof}
    Let's check the axioms:

    \begin{enumerate}
    \item For any \(\vec{x} \in \R^n\), we can use \(B(\vec{x}, \epsilon)\) as a basis element containing \(\vec{x}\).
    \item Now for any \(B_1 = B(\vec{x}_1, \epsilon_1), B_2 = B(\vec{x}_2, \epsilon_2)\), and \(\vec{x} \in B_1 \cap B_2\). then there exists \[B_3 = B\left(\vec{x}, \min \opn |\epsilon_1 - \vec{x_1}|, |\epsilon_2 - \vec{x_2}| \cls\right)\]
    \end{enumerate}
  \end{proof}

  Recall that this is almost exactly how we usually define the usual topology on \(\R^n\), baring minor differences.
\end{exa}

This next proposition will hi-light and explain a commonly used hand-wave. Namely, handling finite unions by claiming that it `follows by induction'.
\begin{prop}
  Let \(X\) be any set, and let \(\mathcal{B}\) be a basis for a topology on \(X\). Then we claim that the ``topology'' \(\tau\) generated by \(\mathcal{B}\) truly is a topology.
\end{prop}
\begin{proof}
  \begin{enumerate}
  \item \(\emptyset\), vacuously open, since there are no points in the empty set.
    \(X\) is open, since for every point \(\vec{x}\), the first basis axiom guarantees that there exists a \(B \in \mathcal{B}\) such that \(\vec{x} \in B\).
  \item Suppose \(U_i \in \tau\) for all \(i \in I\). So let some point \(x \in \bigcup_{i \in I} U_i\). Then there is some \(j \in I\) such that \(x \in U_j\). Since \(U_j\) is open, there exists a \(B \in \mathcal{B}\) such that \(x \in B\) and \(B \subseteq U_j\). So then \(x \in B \subseteq \bigcup_{i \in I}U_i\), so \(\bigcup_{i \in I} U_i\) is open
  \item Let's induct for the final part.

    Suppose \(U, V \in \tau\), and consider \(U \cap V\). Now Let \(x \in U \cap V\). Since \(U\) is open, there is a basis element \(B_1\) such that \(x \in B_1\), and \(B_1 \subseteq U\). Then also since \(V\) is open, \(\exists B_2\) such that \(x \in B_2\) and \(B_2 \subseteq V\). Thus by basis axiom two, \(\exists B_3 \in \mathcal{B}\) such that \(x \in B_3\) and \(B_3 \subseteq B_1 \cap B_2\). So \(U \cap V\) is open. This completes the next part of our induction.

    Suppose that there exists a \(k \in \Z_{>0}\) such that if \(U_1, \ldots U_k \in \tau\), then \(\bigcup_{i = 1}^k U_i \in tau\) for the sake of induction. now suppose that we have some collection \(U_1, \ldots, U_{n+1}\). Then the intersection of the first \(n\) sets will be open, and so we return to just taking an intersection of two open sets. Therefore, we have shown by induction that we can take finite intersections.
  \end{enumerate}
\end{proof}

\begin{exa}
  If \(X = \R^2\) take \[\mathcal{B} = \opn (a, b) \times (c, d) \stbar \begin{matrix} a, b, c, d \in \R \\ a < b, \ c < d \end{matrix} \cls\]

  This is also a basis for a topology. How does it compare to the usual topology, I wonder...
\end{exa}

\begin{exa}
  If \(X\) is any set \[\mathcal{B} = \opn \opn x \cls\stbar x \in X \cls\] is also a basis for a topology. This is (pretty clearly) a basis for the discrete topology.
\end{exa}

\begin{lm}
  Let \(X\) be a set, \(\mathcal{B}\) be a basis for a topology on \(X\), \(\tau\) the topology generated by \(\mathcal{B}\). Then \(\tau\) is the set of all possible unions of the basis elements:
  \begin{align*}
    \tau = \opn \bigcup_{i \in I} B_i\stbar \begin{matrix} I \text{ is any indexing set} \\ B_i \in \mathcal{B} \text{ for all } i \in I \end{matrix} \cls.
  \end{align*}
\end{lm}
\begin{proof}
  Suppose \(U = \bigcup_{i \in I} B_i\), where \(B_i \in \mathcal{B}\). Then if \(x \in U\) there exists \(j \in I\) such that \(x \in B_j\). Then \(x \in B_j \subseteq U\), so \(U\) is open in the generated topology. Conversely, suppose \(U \subseteq X\) is open in the generated topology.

  Then since \(U\) is open for each \(x \in U\) there is a basis element \(B_x \in \mathcal{B}\) such that \(x \in B_x \subseteq U\). Then

  \begin{align*}
    U &= \bigcup_{x \in U} \{x\} \\
      &\subseteq \bigcup_{x \in U} B_x \\
      &\subseteq \bigcup_{x \in U} U = U.
  \end{align*}
  So \(U = \bigcup_{x \in U} B_x\)
\end{proof}

\begin{prop}
  Let \(X\) be a set. Let \(\mathcal{B}\prime, \mathcal{B}\) be bases for topologies on \(X\). Let \(\tau\prime, \tau\) be the respective generated topologies. Then the following are equivalent:
  \begin{enumerate}
  \item \(\tau\prime\) is finer than \(\tau\)
  \item For each \(B \in \mathcal{B}\) and \(x \in B\), there exists \(B\prime \in \mathcal{B}\prime\) such that \(x \in B\prime \subseteq B\). 
  \end{enumerate}
\end{prop}
\begin{proof} \hphantom{1}
  \begin{enumerate}
  \item [(\(1 \Rightarrow 2\))] Assume that \(\tau\prime\) is finer than \(\tau\), and let \(B \in \mathcal{B}\) and \(x \in B\). But then \(B \in \tau\), so \(B \in \tau\prime\). By definition of \(\tau\prime\), since \(B\) is open, and \(x \in B\), \(\exists B\prime \in \mathcal{B}\prime \st x \in B\prime \subseteq B\).
  \item [(\(2 \Rightarrow 1\))] Assume \(\forall B \in \mathcal{B}\), \(x \in B\), \(\exists B\prime \in \mathcal{B}\prime \st x \in B\prime \subseteq B\).

    Now let \(U \in \tau\) be given. Then by definition, for any \(x \in U\), there exists a basis element \(B \in \mathcal{B}\) such that \(x \in B \subseteq U\). Then we apply our hypothesis. for such an \(x\), we have by assumption a \(B\prime \in \mathcal{B}\prime \st x \in B\prime \subseteq B\).

    Then, for all \(x\), \(\exists B\prime \in \mathcal{B}\prime \st x\in B\prime \subseteq U\). So \(U\) is open in \(\tau\prime\). 
  \end{enumerate}
\end{proof}
\end{document}
%%% Local Variables:
%%% mode: latex
%%% TeX-master: t
%%% End:
